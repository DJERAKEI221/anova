% Options for packages loaded elsewhere
\PassOptionsToPackage{unicode}{hyperref}
\PassOptionsToPackage{hyphens}{url}
\PassOptionsToPackage{dvipsnames,svgnames,x11names}{xcolor}
%
\documentclass[
  letterpaper,
]{report}

\usepackage{amsmath,amssymb}
\usepackage{iftex}
\ifPDFTeX
  \usepackage[T1]{fontenc}
  \usepackage[utf8]{inputenc}
  \usepackage{textcomp} % provide euro and other symbols
\else % if luatex or xetex
  \usepackage{unicode-math}
  \defaultfontfeatures{Scale=MatchLowercase}
  \defaultfontfeatures[\rmfamily]{Ligatures=TeX,Scale=1}
\fi
\usepackage{lmodern}
\ifPDFTeX\else  
    % xetex/luatex font selection
\fi
% Use upquote if available, for straight quotes in verbatim environments
\IfFileExists{upquote.sty}{\usepackage{upquote}}{}
\IfFileExists{microtype.sty}{% use microtype if available
  \usepackage[]{microtype}
  \UseMicrotypeSet[protrusion]{basicmath} % disable protrusion for tt fonts
}{}
\makeatletter
\@ifundefined{KOMAClassName}{% if non-KOMA class
  \IfFileExists{parskip.sty}{%
    \usepackage{parskip}
  }{% else
    \setlength{\parindent}{0pt}
    \setlength{\parskip}{6pt plus 2pt minus 1pt}}
}{% if KOMA class
  \KOMAoptions{parskip=half}}
\makeatother
\usepackage{xcolor}
\setlength{\emergencystretch}{3em} % prevent overfull lines
\setcounter{secnumdepth}{5}
% Make \paragraph and \subparagraph free-standing
\makeatletter
\ifx\paragraph\undefined\else
  \let\oldparagraph\paragraph
  \renewcommand{\paragraph}{
    \@ifstar
      \xxxParagraphStar
      \xxxParagraphNoStar
  }
  \newcommand{\xxxParagraphStar}[1]{\oldparagraph*{#1}\mbox{}}
  \newcommand{\xxxParagraphNoStar}[1]{\oldparagraph{#1}\mbox{}}
\fi
\ifx\subparagraph\undefined\else
  \let\oldsubparagraph\subparagraph
  \renewcommand{\subparagraph}{
    \@ifstar
      \xxxSubParagraphStar
      \xxxSubParagraphNoStar
  }
  \newcommand{\xxxSubParagraphStar}[1]{\oldsubparagraph*{#1}\mbox{}}
  \newcommand{\xxxSubParagraphNoStar}[1]{\oldsubparagraph{#1}\mbox{}}
\fi
\makeatother

\usepackage{color}
\usepackage{fancyvrb}
\newcommand{\VerbBar}{|}
\newcommand{\VERB}{\Verb[commandchars=\\\{\}]}
\DefineVerbatimEnvironment{Highlighting}{Verbatim}{commandchars=\\\{\}}
% Add ',fontsize=\small' for more characters per line
\usepackage{framed}
\definecolor{shadecolor}{RGB}{241,243,245}
\newenvironment{Shaded}{\begin{snugshade}}{\end{snugshade}}
\newcommand{\AlertTok}[1]{\textcolor[rgb]{0.68,0.00,0.00}{#1}}
\newcommand{\AnnotationTok}[1]{\textcolor[rgb]{0.37,0.37,0.37}{#1}}
\newcommand{\AttributeTok}[1]{\textcolor[rgb]{0.40,0.45,0.13}{#1}}
\newcommand{\BaseNTok}[1]{\textcolor[rgb]{0.68,0.00,0.00}{#1}}
\newcommand{\BuiltInTok}[1]{\textcolor[rgb]{0.00,0.23,0.31}{#1}}
\newcommand{\CharTok}[1]{\textcolor[rgb]{0.13,0.47,0.30}{#1}}
\newcommand{\CommentTok}[1]{\textcolor[rgb]{0.37,0.37,0.37}{#1}}
\newcommand{\CommentVarTok}[1]{\textcolor[rgb]{0.37,0.37,0.37}{\textit{#1}}}
\newcommand{\ConstantTok}[1]{\textcolor[rgb]{0.56,0.35,0.01}{#1}}
\newcommand{\ControlFlowTok}[1]{\textcolor[rgb]{0.00,0.23,0.31}{\textbf{#1}}}
\newcommand{\DataTypeTok}[1]{\textcolor[rgb]{0.68,0.00,0.00}{#1}}
\newcommand{\DecValTok}[1]{\textcolor[rgb]{0.68,0.00,0.00}{#1}}
\newcommand{\DocumentationTok}[1]{\textcolor[rgb]{0.37,0.37,0.37}{\textit{#1}}}
\newcommand{\ErrorTok}[1]{\textcolor[rgb]{0.68,0.00,0.00}{#1}}
\newcommand{\ExtensionTok}[1]{\textcolor[rgb]{0.00,0.23,0.31}{#1}}
\newcommand{\FloatTok}[1]{\textcolor[rgb]{0.68,0.00,0.00}{#1}}
\newcommand{\FunctionTok}[1]{\textcolor[rgb]{0.28,0.35,0.67}{#1}}
\newcommand{\ImportTok}[1]{\textcolor[rgb]{0.00,0.46,0.62}{#1}}
\newcommand{\InformationTok}[1]{\textcolor[rgb]{0.37,0.37,0.37}{#1}}
\newcommand{\KeywordTok}[1]{\textcolor[rgb]{0.00,0.23,0.31}{\textbf{#1}}}
\newcommand{\NormalTok}[1]{\textcolor[rgb]{0.00,0.23,0.31}{#1}}
\newcommand{\OperatorTok}[1]{\textcolor[rgb]{0.37,0.37,0.37}{#1}}
\newcommand{\OtherTok}[1]{\textcolor[rgb]{0.00,0.23,0.31}{#1}}
\newcommand{\PreprocessorTok}[1]{\textcolor[rgb]{0.68,0.00,0.00}{#1}}
\newcommand{\RegionMarkerTok}[1]{\textcolor[rgb]{0.00,0.23,0.31}{#1}}
\newcommand{\SpecialCharTok}[1]{\textcolor[rgb]{0.37,0.37,0.37}{#1}}
\newcommand{\SpecialStringTok}[1]{\textcolor[rgb]{0.13,0.47,0.30}{#1}}
\newcommand{\StringTok}[1]{\textcolor[rgb]{0.13,0.47,0.30}{#1}}
\newcommand{\VariableTok}[1]{\textcolor[rgb]{0.07,0.07,0.07}{#1}}
\newcommand{\VerbatimStringTok}[1]{\textcolor[rgb]{0.13,0.47,0.30}{#1}}
\newcommand{\WarningTok}[1]{\textcolor[rgb]{0.37,0.37,0.37}{\textit{#1}}}

\providecommand{\tightlist}{%
  \setlength{\itemsep}{0pt}\setlength{\parskip}{0pt}}\usepackage{longtable,booktabs,array}
\usepackage{calc} % for calculating minipage widths
% Correct order of tables after \paragraph or \subparagraph
\usepackage{etoolbox}
\makeatletter
\patchcmd\longtable{\par}{\if@noskipsec\mbox{}\fi\par}{}{}
\makeatother
% Allow footnotes in longtable head/foot
\IfFileExists{footnotehyper.sty}{\usepackage{footnotehyper}}{\usepackage{footnote}}
\makesavenoteenv{longtable}
\usepackage{graphicx}
\makeatletter
\def\maxwidth{\ifdim\Gin@nat@width>\linewidth\linewidth\else\Gin@nat@width\fi}
\def\maxheight{\ifdim\Gin@nat@height>\textheight\textheight\else\Gin@nat@height\fi}
\makeatother
% Scale images if necessary, so that they will not overflow the page
% margins by default, and it is still possible to overwrite the defaults
% using explicit options in \includegraphics[width, height, ...]{}
\setkeys{Gin}{width=\maxwidth,height=\maxheight,keepaspectratio}
% Set default figure placement to htbp
\makeatletter
\def\fps@figure{htbp}
\makeatother

\makeatletter
\@ifpackageloaded{tcolorbox}{}{\usepackage[skins,breakable]{tcolorbox}}
\@ifpackageloaded{fontawesome5}{}{\usepackage{fontawesome5}}
\definecolor{quarto-callout-color}{HTML}{909090}
\definecolor{quarto-callout-note-color}{HTML}{0758E5}
\definecolor{quarto-callout-important-color}{HTML}{CC1914}
\definecolor{quarto-callout-warning-color}{HTML}{EB9113}
\definecolor{quarto-callout-tip-color}{HTML}{00A047}
\definecolor{quarto-callout-caution-color}{HTML}{FC5300}
\definecolor{quarto-callout-color-frame}{HTML}{acacac}
\definecolor{quarto-callout-note-color-frame}{HTML}{4582ec}
\definecolor{quarto-callout-important-color-frame}{HTML}{d9534f}
\definecolor{quarto-callout-warning-color-frame}{HTML}{f0ad4e}
\definecolor{quarto-callout-tip-color-frame}{HTML}{02b875}
\definecolor{quarto-callout-caution-color-frame}{HTML}{fd7e14}
\makeatother
\makeatletter
\@ifpackageloaded{bookmark}{}{\usepackage{bookmark}}
\makeatother
\makeatletter
\@ifpackageloaded{caption}{}{\usepackage{caption}}
\AtBeginDocument{%
\ifdefined\contentsname
  \renewcommand*\contentsname{Table of contents}
\else
  \newcommand\contentsname{Table of contents}
\fi
\ifdefined\listfigurename
  \renewcommand*\listfigurename{List of Figures}
\else
  \newcommand\listfigurename{List of Figures}
\fi
\ifdefined\listtablename
  \renewcommand*\listtablename{List of Tables}
\else
  \newcommand\listtablename{List of Tables}
\fi
\ifdefined\figurename
  \renewcommand*\figurename{Figure}
\else
  \newcommand\figurename{Figure}
\fi
\ifdefined\tablename
  \renewcommand*\tablename{Table}
\else
  \newcommand\tablename{Table}
\fi
}
\@ifpackageloaded{float}{}{\usepackage{float}}
\floatstyle{ruled}
\@ifundefined{c@chapter}{\newfloat{codelisting}{h}{lop}}{\newfloat{codelisting}{h}{lop}[chapter]}
\floatname{codelisting}{Listing}
\newcommand*\listoflistings{\listof{codelisting}{List of Listings}}
\makeatother
\makeatletter
\makeatother
\makeatletter
\@ifpackageloaded{caption}{}{\usepackage{caption}}
\@ifpackageloaded{subcaption}{}{\usepackage{subcaption}}
\makeatother

\ifLuaTeX
  \usepackage{selnolig}  % disable illegal ligatures
\fi
\usepackage{bookmark}

\IfFileExists{xurl.sty}{\usepackage{xurl}}{} % add URL line breaks if available
\urlstyle{same} % disable monospaced font for URLs
\hypersetup{
  pdftitle={TRAVAIL PRATIQUE D'ANOVA},
  colorlinks=true,
  linkcolor={blue},
  filecolor={Maroon},
  citecolor={Blue},
  urlcolor={Blue},
  pdfcreator={LaTeX via pandoc}}


\title{TRAVAIL PRATIQUE D'ANOVA}
\author{}
\date{}

\begin{document}
\maketitle

\renewcommand*\contentsname{Table of contents}
{
\hypersetup{linkcolor=}
\setcounter{tocdepth}{2}
\tableofcontents
}

\bookmarksetup{startatroot}

\chapter{Bienvenue}\label{bienvenue}

\textbf{Agence nationale de la Statistique et de la Démographie (ANSD)}

\begin{figure}[H]

{\centering \includegraphics[width=1.45833in,height=\textheight]{images/ansd.png}

}

\caption{ANSD}

\end{figure}%

École nationale de la Statistique et de l'Analyse économique Pierre
NDIAYE (ENSAE)

\begin{figure}[H]

{\centering \includegraphics[width=1.45833in,height=\textheight]{images/ensae.png}

}

\caption{ENSAE}

\end{figure}%

TRAVAIL PRATIQUE D'ANOVA

Analyse des effets de la variété et du type de fertilisant sur la
croissance des plantes

\textbf{Rédigé par :} Mistalengar Yves DJERAKEI, Moussa DIÉMÉ, Compaoré
BASSIROU\\
\emph{Élèves Ingénieurs Statisticiens Économistes (ISE)}

\textbf{Sous l'encadrement de :} M. Carlos AKAKPOVI, Ingénieur
Statisticien Économiste (ISE)

\begin{center}\rule{0.5\linewidth}{0.5pt}\end{center}

\bookmarksetup{startatroot}

\chapter*{Bienvenue}\label{bienvenue-1}
\addcontentsline{toc}{chapter}{Bienvenue}

\markboth{Bienvenue}{Bienvenue}

Ce livre présente une \textbf{analyse de la variance à mesures répétées}
appliquée à l'étude de la \textbf{croissance des plantes} selon le type
de terreau et la variété, à travers le temps.

\section{Accès aux chapitres}\label{accuxe8s-aux-chapitres}

Les trois chapitres sont accessibles via la \textbf{table des matières}
(barre latérale gauche) ou directement ci-dessous :

\begin{itemize}
\item
  \href{01-theorie-anova-mesures-repetees.qmd}{\textbf{Chapitre 1 :
  Théorie sur l'ANOVA à mesures répétées}}\\
  Contexte, définition, hypothèses (normalité, homogénéité, sphéricité),
  test de Mauchly et modèle mixte (split-plot).
\item
  \href{02-analyse-descriptive.qmd}{\textbf{Chapitre 2 : Analyse
  descriptive des variables d'étude}}\\
  Import des données, tableau des moyennes et écarts-types, graphique
  d'évolution (taille selon la période, par groupe et variété).
\item
  \href{03-facteurs-explicatifs.qmd}{\textbf{Chapitre 3 : Facteurs
  explicatifs de la variable dépendante}}\\
  Un facteur ; deux facteurs avec ou sans interaction ; modèle ANOVA
  mixte, vérifications (Levene, Mauchly), comparaisons post-hoc et
  conclusion.
\end{itemize}

\bookmarksetup{startatroot}

\chapter{Théorie sur l'Analyse de la Variance à mesures
répétées}\label{thuxe9orie-sur-lanalyse-de-la-variance-uxe0-mesures-ruxe9puxe9tuxe9es}

Cadre conceptuel, hypothèses et modèles

\hfill\break

\begin{tcolorbox}[enhanced jigsaw, opacitybacktitle=0.6, colback=white, colbacktitle=quarto-callout-note-color!10!white, leftrule=.75mm, coltitle=black, colframe=quarto-callout-note-color-frame, opacityback=0, left=2mm, rightrule=.15mm, toptitle=1mm, title=\textcolor{quarto-callout-note-color}{\faInfo}\hspace{0.5em}{Idée clé}, arc=.35mm, bottomtitle=1mm, toprule=.15mm, breakable, bottomrule=.15mm, titlerule=0mm]

L'ANOVA à mesures répétées est la méthode adaptée lorsque les
\textbf{mêmes sujets} sont mesurés à plusieurs reprises (temps ou
conditions) : elle modélise explicitement la \textbf{corrélation
intra-sujet} et évite les biais de l'ANOVA classique.

\end{tcolorbox}

\bookmarksetup{startatroot}

\chapter{Contexte et limite de l'ANOVA
classique}\label{contexte-et-limite-de-lanova-classique}

Lorsque les \textbf{mêmes individus} sont mesurés à plusieurs reprises
(dans le temps ou sous différentes conditions), l'ANOVA classique
devient inadaptée : elle repose sur l'\textbf{indépendance des
observations}, alors que les mesures répétées sur un même sujet
introduisent une \textbf{dépendance statistique} (corrélation
intra-sujet). Si cette structure n'est pas prise en compte, la variance
est sous-estimée et les conclusions peuvent être biaisées (risque
d'erreur de type I augmenté).

\bookmarksetup{startatroot}

\chapter{Définition et caractéristiques de l'ANOVA à mesures
répétées}\label{duxe9finition-et-caractuxe9ristiques-de-lanova-uxe0-mesures-ruxe9puxe9tuxe9es}

L'\textbf{ANOVA à mesures répétées} (ANOVA-MR) est une extension de
l'ANOVA conçue pour des données où les mêmes individus sont mesurés
plusieurs fois. Elle permet de \textbf{comparer les moyennes} d'une
variable quantitative à différents moments ou sous différentes
conditions, en tenant compte \textbf{explicitement de la corrélation}
entre mesures répétées.

\textbf{Caractéristiques clés :}

\begin{itemize}
\tightlist
\item
  \textbf{Facteur intra-sujet} : le facteur (temps, condition) varie à
  l'intérieur de chaque sujet ; chaque sujet est observé pour toutes les
  modalités.
\item
  \textbf{Contrôle de la variabilité inter-sujets} : on sépare la
  variance due aux différences entre individus (effet sujet) de la
  variance intra-sujet.
\item
  \textbf{Modélisation de la covariance} : la structure de corrélation
  entre mesures est prise en compte dans le modèle.
\item
  \textbf{Plus grande puissance} : pour un même nombre de sujets, la
  méthode améliore la capacité à détecter un effet réel du facteur
  intra-sujet.
\end{itemize}

On distingue notamment : (1) l'ANOVA à \textbf{un facteur intra-sujet}
(ex. évolution dans le temps pour tous les sujets) ; (2) le
\textbf{modèle mixte (split-plot)}, avec au moins un facteur
inter-sujets (ex. groupe) et un facteur intra-sujets (ex. période). Dans
ce travail pratique, on se concentre sur ces deux types de plans, qui
couvrent l'essentiel des applications courantes.

\bookmarksetup{startatroot}

\chapter{Position du problème et structure des
données}\label{position-du-probluxe8me-et-structure-des-donnuxe9es}

On dispose de \(n\) sujets et de \(t\) temps (ou conditions). La mesure
pour le sujet \(i\) au temps \(j\) est notée \(Y_{ij}\). La
\textbf{variabilité totale} se décompose en trois composantes :

\[SCT = SS_{\text{sujets}} + SS_{\text{temps}} + SS_{\text{erreur}}\]

\begin{itemize}
\tightlist
\item
  \textbf{\(SS_{\text{sujets}}\)} : variabilité inter-individuelle
  (différences entre sujets).
\item
  \textbf{\(SS_{\text{temps}}\)} : effet du facteur intra-sujet
  (évolution dans le temps ou entre conditions).
\item
  \textbf{\(SS_{\text{erreur}}\)} : interaction sujet × temps
  (résiduelle).
\end{itemize}

Les \textbf{hypothèses statistiques} testées sont : \(H_0\) :
\(\mu_1 = \mu_2 = \cdots = \mu_t\) (aucun effet du facteur
temps/condition) contre \(H_1\) : il existe au moins deux niveaux dont
les moyennes diffèrent.

\bookmarksetup{startatroot}

\chapter{Hypothèses du modèle et conditions de
validité}\label{hypothuxe8ses-du-moduxe8le-et-conditions-de-validituxe9}

Pour que les tests soient valides, les hypothèses suivantes doivent être
vérifiées :

\begin{enumerate}
\def\labelenumi{\arabic{enumi}.}
\item
  \textbf{Normalité} : pour chaque temps (ou condition), les résidus du
  modèle sont supposés distribués normalement. On peut l'évaluer par des
  graphiques (Q-Q plot) ou des tests (Shapiro-Wilk sur les résidus).
\item
  \textbf{Homogénéité des variances} (facteurs inter-sujets) : lorsqu'il
  existe des groupes (ex. terreau, variété), les variances des erreurs
  doivent être égales entre groupes. Le test de \textbf{Levene} est
  utilisé pour cette vérification.
\item
  \textbf{Sphéricité} (facteur intra-sujet) : les variances des
  \textbf{différences} entre toutes les paires de temps doivent être
  égales : \(\text{Var}(Y_{ij} - Y_{ik})\) constante pour toutes les
  paires \((j,k)\). C'est l'hypothèse testée par le \textbf{test de
  Mauchly}. Si elle est rejetée (\(p < 0{,}05\)), on applique une
  \textbf{correction des degrés de liberté} (Greenhouse-Geisser ou
  Huynh-Feldt) pour garder des conclusions valides.
\end{enumerate}

\begin{tcolorbox}[enhanced jigsaw, opacitybacktitle=0.6, colback=white, colbacktitle=quarto-callout-tip-color!10!white, leftrule=.75mm, coltitle=black, colframe=quarto-callout-tip-color-frame, opacityback=0, left=2mm, rightrule=.15mm, toptitle=1mm, title=\textcolor{quarto-callout-tip-color}{\faLightbulb}\hspace{0.5em}{Ordre recommandé}, arc=.35mm, bottomtitle=1mm, toprule=.15mm, breakable, bottomrule=.15mm, titlerule=0mm]

\begin{enumerate}
\def\labelenumi{\arabic{enumi}.}
\tightlist
\item
  Vérifier la \textbf{normalité} des résidus (Q-Q plot, Shapiro-Wilk).\\
\item
  Effectuer le \textbf{test de Mauchly} (sphéricité).\\
\item
  En cas de rejet (\(p < 0{,}05\)), utiliser les \textbf{ddl corrigés}
  (Greenhouse-Geisser ou Huynh-Feldt).\\
\item
  Réaliser l'\textbf{ANOVA-MR} et les comparaisons post-hoc si
  nécessaire.
\end{enumerate}

\end{tcolorbox}

\bookmarksetup{startatroot}

\chapter{Test de Mauchly et
corrections}\label{test-de-mauchly-et-corrections}

Le \textbf{test de Mauchly} évalue si la matrice de covariance des
différences entre niveaux du facteur intra-sujet a une forme sphérique.
Sous \(H_0\), une transformation de la statistique \(W\) suit
approximativement une loi du \(\chi^2\). Si \(p < \alpha\) (souvent
0,05), on rejette l'hypothèse de sphéricité et on rapporte les résultats
avec les \textbf{corrections de Greenhouse-Geisser} ou
\textbf{Huynh-Feldt} sur les degrés de liberté (et donc sur la
\(p\)-valeur).

\bookmarksetup{startatroot}

\chapter{Modèle mixte (split-plot) et plans
factoriels}\label{moduxe8le-mixte-split-plot-et-plans-factoriels}

Un plan \textbf{mixte} combine un \textbf{facteur inter-sujets} (ex.
groupe, variété) et un \textbf{facteur intra-sujets} (ex. période). Le
modèle inclut les effets principaux de ces facteurs, leur
\textbf{interaction} (l'évolution dans le temps peut différer selon le
groupe), et un effet aléatoire \textbf{sujet} (ou ID). Chaque effet fixe
est testé contre le terme d'erreur approprié (inter-sujets ou
intra-sujets). En cas d'\textbf{interaction significative}, on procède à
des analyses des effets simples et à des comparaisons multiples
(post-hoc, ex. Tukey ou Bonferroni).

Cette approche permet d'étudier successivement l'effet d'\textbf{un seul
facteur} sur la variable dépendante, puis l'effet de \textbf{deux
facteurs} (avec ou sans interaction), en choisissant le terme d'erreur
adapté à chaque effet.

\bookmarksetup{startatroot}

\chapter{Lien avec les chapitres
suivants}\label{lien-avec-les-chapitres-suivants}

Les notions présentées ici (facteurs inter- et intra-sujets, hypothèses
de normalité, homogénéité, sphéricité, test de Mauchly, corrections de
Greenhouse--Geisser et Huynh--Feldt, modèles mixtes) sont mises en œuvre
dans les deux chapitres appliqués. Le \textbf{chapitre 2} décrit les
données de hauteur de plantes et explore graphiquement les relations
entre fertilisant, variété et période. Le \textbf{chapitre 3} applique
ensuite une ANOVA à mesures répétées sur ces données, en vérifiant les
hypothèses et en interprétant les effets principaux, les interactions et
les comparaisons post-hoc à la lumière du cadre théorique établi ici.

\bookmarksetup{startatroot}

\chapter{Analyse descriptive des variables
d'étude}\label{analyse-descriptive-des-variables-duxe9tude}

Données, moyennes et graphiques d'évolution

\hfill\break

Ce chapitre présente l'import des données de croissance, les
statistiques descriptives (moyennes, écarts-types, effectifs) et
plusieurs graphiques pour décrire la \textbf{hauteur} selon la période,
le fertilisant (terreau) et la variété.

\bookmarksetup{startatroot}

\chapter{Import des données et
préparation}\label{import-des-donnuxe9es-et-pruxe9paration}

Le fichier \textbf{\texttt{donnees/donnees.xlsx}} est lu ; les variables
sont \emph{fertilisant}, \emph{variete}, \emph{periode}, \emph{hauteur}.
Plan : mesures répétées \textbf{T1--T4 pour Var1 et Var2} (et pour
chaque fertilisant). Le facteur \textbf{periode} a pour modalités T1,
T2, T3, T4. L'identifiant sujet (\emph{id}) est créé si absent.

\begin{Shaded}
\begin{Highlighting}[]
\FunctionTok{library}\NormalTok{(tidyverse)}
\FunctionTok{library}\NormalTok{(knitr)}
\ControlFlowTok{if}\NormalTok{ (}\SpecialCharTok{!}\FunctionTok{requireNamespace}\NormalTok{(}\StringTok{"gtsummary"}\NormalTok{, }\AttributeTok{quietly =} \ConstantTok{TRUE}\NormalTok{)) }\FunctionTok{install.packages}\NormalTok{(}\StringTok{"gtsummary"}\NormalTok{, }\AttributeTok{repos =} \StringTok{"https://cloud.r{-}project.org"}\NormalTok{, }\AttributeTok{quiet =} \ConstantTok{TRUE}\NormalTok{)}
\FunctionTok{library}\NormalTok{(gtsummary)}
\end{Highlighting}
\end{Shaded}

\begin{Shaded}
\begin{Highlighting}[]
\CommentTok{\# {-}{-}{-} Import : plan = mesures répétées T1–T4 pour Var1 et Var2 (colonnes : fertilisant, variete, periode, hauteur) {-}{-}{-}}
\ControlFlowTok{if}\NormalTok{ (}\SpecialCharTok{!}\FunctionTok{requireNamespace}\NormalTok{(}\StringTok{"readxl"}\NormalTok{, }\AttributeTok{quietly =} \ConstantTok{TRUE}\NormalTok{)) }\FunctionTok{install.packages}\NormalTok{(}\StringTok{"readxl"}\NormalTok{, }\AttributeTok{repos =} \StringTok{"https://cloud.r{-}project.org"}\NormalTok{, }\AttributeTok{quiet =} \ConstantTok{TRUE}\NormalTok{)}
\FunctionTok{library}\NormalTok{(readxl)}
\NormalTok{d }\OtherTok{\textless{}{-}} \FunctionTok{read\_excel}\NormalTok{(}\StringTok{"donnees/donnees.xlsx"}\NormalTok{)}

\CommentTok{\# Hauteur en numérique (virgule décimale gérée)}
\NormalTok{d}\SpecialCharTok{$}\NormalTok{hauteur }\OtherTok{\textless{}{-}} \FunctionTok{as.numeric}\NormalTok{(}\FunctionTok{gsub}\NormalTok{(}\StringTok{","}\NormalTok{, }\StringTok{"."}\NormalTok{, }\FunctionTok{as.character}\NormalTok{(d}\SpecialCharTok{$}\NormalTok{hauteur)))}

\CommentTok{\# Identifiant sujet : un id par plante (même id pour les 4 périodes d\textquotesingle{}une même plante)}
\ControlFlowTok{if}\NormalTok{ (}\SpecialCharTok{!}\StringTok{"id"} \SpecialCharTok{\%in\%} \FunctionTok{names}\NormalTok{(d)) \{}
\NormalTok{  d }\OtherTok{\textless{}{-}}\NormalTok{ d }\SpecialCharTok{\%\textgreater{}\%}
    \FunctionTok{arrange}\NormalTok{(fertilisant, variete, periode) }\SpecialCharTok{\%\textgreater{}\%}
    \FunctionTok{group\_by}\NormalTok{(fertilisant, variete, periode) }\SpecialCharTok{\%\textgreater{}\%}
    \FunctionTok{mutate}\NormalTok{(}\AttributeTok{within\_subj =} \FunctionTok{row\_number}\NormalTok{()) }\SpecialCharTok{\%\textgreater{}\%}
    \FunctionTok{ungroup}\NormalTok{() }\SpecialCharTok{\%\textgreater{}\%}
    \FunctionTok{mutate}\NormalTok{(}\AttributeTok{id =} \FunctionTok{paste0}\NormalTok{(}\FunctionTok{as.character}\NormalTok{(fertilisant), }\StringTok{"\_"}\NormalTok{, }\FunctionTok{as.character}\NormalTok{(variete), }\StringTok{"\_"}\NormalTok{, within\_subj))}
\NormalTok{\}}

\CommentTok{\# Facteurs}
\NormalTok{d }\OtherTok{\textless{}{-}}\NormalTok{ d }\SpecialCharTok{\%\textgreater{}\%}
  \FunctionTok{mutate}\NormalTok{(}
    \AttributeTok{periode     =} \FunctionTok{factor}\NormalTok{(}\FunctionTok{as.character}\NormalTok{(periode), }\AttributeTok{levels =} \FunctionTok{c}\NormalTok{(}\StringTok{"T1"}\NormalTok{, }\StringTok{"T2"}\NormalTok{, }\StringTok{"T3"}\NormalTok{, }\StringTok{"T4"}\NormalTok{)),}
    \AttributeTok{fertilisant =} \FunctionTok{factor}\NormalTok{(fertilisant, }\AttributeTok{levels =} \FunctionTok{c}\NormalTok{(}\StringTok{"Ma"}\NormalTok{, }\StringTok{"Ca"}\NormalTok{, }\StringTok{"An"}\NormalTok{)),}
    \AttributeTok{variete     =} \FunctionTok{factor}\NormalTok{(variete),}
    \AttributeTok{id          =} \FunctionTok{factor}\NormalTok{(id)}
\NormalTok{  )}
\end{Highlighting}
\end{Shaded}

\begin{Shaded}
\begin{Highlighting}[]
\CommentTok{\# {-}{-}{-} Aperçu du jeu de données (format long) {-}{-}{-}}
\NormalTok{d }\SpecialCharTok{\%\textgreater{}\%}
  \FunctionTok{tbl\_summary}\NormalTok{(}\AttributeTok{include =} \FunctionTok{c}\NormalTok{(fertilisant, variete, periode, hauteur), }\AttributeTok{statistic =} \FunctionTok{list}\NormalTok{(hauteur }\SpecialCharTok{\textasciitilde{}} \StringTok{"\{mean\} (\{sd\})"}\NormalTok{), }\AttributeTok{label =} \FunctionTok{list}\NormalTok{(hauteur }\SpecialCharTok{\textasciitilde{}} \StringTok{"Hauteur"}\NormalTok{)) }\SpecialCharTok{\%\textgreater{}\%}
  \FunctionTok{modify\_caption}\NormalTok{(}\StringTok{"Résumé descriptif des variables (fertilisant, variété, période, hauteur)"}\NormalTok{)}
\end{Highlighting}
\end{Shaded}

\begin{longtable}[]{@{}lc@{}}
\caption{Résumé descriptif des variables (fertilisant, variété, période,
hauteur)}\tabularnewline
\toprule\noalign{}
\textbf{Characteristic} & \textbf{N = 253} \\
\midrule\noalign{}
\endfirsthead
\toprule\noalign{}
\textbf{Characteristic} & \textbf{N = 253} \\
\midrule\noalign{}
\endhead
\bottomrule\noalign{}
\endlastfoot
fertilisant & \\
Ma & 102 (40\%) \\
Ca & 73 (29\%) \\
An & 78 (31\%) \\
variete & \\
Var1 & 129 (51\%) \\
Var2 & 124 (49\%) \\
periode & \\
T1 & 78 (31\%) \\
T2 & 57 (23\%) \\
T3 & 63 (25\%) \\
T4 & 55 (22\%) \\
Hauteur & 34 (8) \\
\end{longtable}

\bookmarksetup{startatroot}

\chapter{Structure des données}\label{structure-des-donnuxe9es}

Nombre total d'observations, de sujets (plantes) et répartition par
fertilisant et variété.

\begin{Shaded}
\begin{Highlighting}[]
\CommentTok{\# {-}{-}{-} Effectifs : nombre d\textquotesingle{}observations et de sujets {-}{-}{-}}
\NormalTok{n\_obs   }\OtherTok{\textless{}{-}} \FunctionTok{nrow}\NormalTok{(d)}
\NormalTok{n\_sujets }\OtherTok{\textless{}{-}} \FunctionTok{n\_distinct}\NormalTok{(d}\SpecialCharTok{$}\NormalTok{id)}
\CommentTok{\# Répartition par fertilisant et variété (nombre de sujets uniques)}
\NormalTok{sujets\_uniq }\OtherTok{\textless{}{-}}\NormalTok{ d }\SpecialCharTok{\%\textgreater{}\%} \FunctionTok{distinct}\NormalTok{(id, fertilisant, variete)}

\NormalTok{sujets\_uniq }\SpecialCharTok{\%\textgreater{}\%}
  \FunctionTok{tbl\_cross}\NormalTok{(}\AttributeTok{row =}\NormalTok{ fertilisant, }\AttributeTok{col =}\NormalTok{ variete, }\AttributeTok{margin =} \StringTok{"row"}\NormalTok{, }\AttributeTok{percent =} \StringTok{"none"}\NormalTok{) }\SpecialCharTok{\%\textgreater{}\%}
  \FunctionTok{modify\_caption}\NormalTok{(}\StringTok{"Nombre de sujets (plantes) par fertilisant (terreau) et variété"}\NormalTok{)}
\end{Highlighting}
\end{Shaded}

\begin{longtable}[]{@{}lcc@{}}
\caption{Nombre de sujets (plantes) par fertilisant (terreau) et
variété}\tabularnewline
\toprule\noalign{}
& Var1 & Var2 \\
\midrule\noalign{}
\endfirsthead
\toprule\noalign{}
& Var1 & Var2 \\
\midrule\noalign{}
\endhead
\bottomrule\noalign{}
\endlastfoot
fertilisant & & \\
Ma & 18 & 16 \\
Ca & 12 & 11 \\
An & 11 & 13 \\
Total & 41 & 40 \\
\end{longtable}

\begin{Shaded}
\begin{Highlighting}[]
\FunctionTok{cat}\NormalTok{(}\StringTok{"}\SpecialCharTok{\textbackslash{}n}\StringTok{**Total :**"}\NormalTok{, n\_obs, }\StringTok{"observations,"}\NormalTok{, n\_sujets, }\StringTok{"sujets.}\SpecialCharTok{\textbackslash{}n}\StringTok{"}\NormalTok{)}
\end{Highlighting}
\end{Shaded}

\begin{verbatim}

**Total :** 253 observations, 81 sujets.
\end{verbatim}

\bookmarksetup{startatroot}

\chapter{Résumé de la variable
hauteur}\label{ruxe9sumuxe9-de-la-variable-hauteur}

Statistiques globales (moyenne, écart-type, min, max) et par facteur.

\begin{Shaded}
\begin{Highlighting}[]
\CommentTok{\# {-}{-}{-} Résumé global de la variable hauteur {-}{-}{-}}
\NormalTok{hauteur\_num }\OtherTok{\textless{}{-}} \FunctionTok{as.numeric}\NormalTok{(}\FunctionTok{as.character}\NormalTok{(d}\SpecialCharTok{$}\NormalTok{hauteur))}
\NormalTok{resum\_global }\OtherTok{\textless{}{-}} \FunctionTok{tibble}\NormalTok{(}
  \AttributeTok{Variable =} \StringTok{"Hauteur"}\NormalTok{,}
  \AttributeTok{n =} \FunctionTok{sum}\NormalTok{(}\SpecialCharTok{!}\FunctionTok{is.na}\NormalTok{(hauteur\_num)),}
  \StringTok{\textasciigrave{}}\AttributeTok{Moyenne (ET)}\StringTok{\textasciigrave{}} \OtherTok{=} \FunctionTok{sprintf}\NormalTok{(}\StringTok{"\%.2f (\%.2f)"}\NormalTok{, }\FunctionTok{mean}\NormalTok{(hauteur\_num, }\AttributeTok{na.rm =} \ConstantTok{TRUE}\NormalTok{), }\FunctionTok{sd}\NormalTok{(hauteur\_num, }\AttributeTok{na.rm =} \ConstantTok{TRUE}\NormalTok{)),}
  \AttributeTok{Min =} \FunctionTok{sprintf}\NormalTok{(}\StringTok{"\%.2f"}\NormalTok{, }\FunctionTok{min}\NormalTok{(hauteur\_num, }\AttributeTok{na.rm =} \ConstantTok{TRUE}\NormalTok{)),}
  \AttributeTok{Max =} \FunctionTok{sprintf}\NormalTok{(}\StringTok{"\%.2f"}\NormalTok{, }\FunctionTok{max}\NormalTok{(hauteur\_num, }\AttributeTok{na.rm =} \ConstantTok{TRUE}\NormalTok{))}
\NormalTok{)}
\NormalTok{knitr}\SpecialCharTok{::}\FunctionTok{kable}\NormalTok{(resum\_global, }\AttributeTok{caption =} \StringTok{"Résumé global de la variable hauteur"}\NormalTok{)}
\end{Highlighting}
\end{Shaded}

\begin{longtable}[]{@{}lrlll@{}}
\caption{Résumé global de la variable hauteur}\tabularnewline
\toprule\noalign{}
Variable & n & Moyenne (ET) & Min & Max \\
\midrule\noalign{}
\endfirsthead
\toprule\noalign{}
Variable & n & Moyenne (ET) & Min & Max \\
\midrule\noalign{}
\endhead
\bottomrule\noalign{}
\endlastfoot
Hauteur & 253 & 34.27 (8.16) & 8.50 & 56.20 \\
\end{longtable}

\begin{Shaded}
\begin{Highlighting}[]
\CommentTok{\# {-}{-}{-} Résumé de la hauteur par fertilisant, par variété et par période {-}{-}{-}}
\NormalTok{resum\_grp }\OtherTok{\textless{}{-}}\NormalTok{ d }\SpecialCharTok{\%\textgreater{}\%} \FunctionTok{group\_by}\NormalTok{(fertilisant) }\SpecialCharTok{\%\textgreater{}\%} \FunctionTok{summarise}\NormalTok{(}\AttributeTok{n =} \FunctionTok{n}\NormalTok{(), }\StringTok{\textasciigrave{}}\AttributeTok{Moyenne (ET)}\StringTok{\textasciigrave{}} \OtherTok{=} \FunctionTok{sprintf}\NormalTok{(}\StringTok{"\%.2f (\%.2f)"}\NormalTok{, }\FunctionTok{mean}\NormalTok{(hauteur, }\AttributeTok{na.rm =} \ConstantTok{TRUE}\NormalTok{), }\FunctionTok{sd}\NormalTok{(hauteur, }\AttributeTok{na.rm =} \ConstantTok{TRUE}\NormalTok{)), }\AttributeTok{.groups =} \StringTok{"drop"}\NormalTok{)}
\NormalTok{resum\_var }\OtherTok{\textless{}{-}}\NormalTok{ d }\SpecialCharTok{\%\textgreater{}\%} \FunctionTok{group\_by}\NormalTok{(variete) }\SpecialCharTok{\%\textgreater{}\%} \FunctionTok{summarise}\NormalTok{(}\AttributeTok{n =} \FunctionTok{n}\NormalTok{(), }\StringTok{\textasciigrave{}}\AttributeTok{Moyenne (ET)}\StringTok{\textasciigrave{}} \OtherTok{=} \FunctionTok{sprintf}\NormalTok{(}\StringTok{"\%.2f (\%.2f)"}\NormalTok{, }\FunctionTok{mean}\NormalTok{(hauteur, }\AttributeTok{na.rm =} \ConstantTok{TRUE}\NormalTok{), }\FunctionTok{sd}\NormalTok{(hauteur, }\AttributeTok{na.rm =} \ConstantTok{TRUE}\NormalTok{)), }\AttributeTok{.groups =} \StringTok{"drop"}\NormalTok{)}
\NormalTok{resum\_per }\OtherTok{\textless{}{-}}\NormalTok{ d }\SpecialCharTok{\%\textgreater{}\%} \FunctionTok{group\_by}\NormalTok{(periode) }\SpecialCharTok{\%\textgreater{}\%} \FunctionTok{summarise}\NormalTok{(}\AttributeTok{n =} \FunctionTok{n}\NormalTok{(), }\StringTok{\textasciigrave{}}\AttributeTok{Moyenne (ET)}\StringTok{\textasciigrave{}} \OtherTok{=} \FunctionTok{sprintf}\NormalTok{(}\StringTok{"\%.2f (\%.2f)"}\NormalTok{, }\FunctionTok{mean}\NormalTok{(hauteur, }\AttributeTok{na.rm =} \ConstantTok{TRUE}\NormalTok{), }\FunctionTok{sd}\NormalTok{(hauteur, }\AttributeTok{na.rm =} \ConstantTok{TRUE}\NormalTok{)), }\AttributeTok{.groups =} \StringTok{"drop"}\NormalTok{)}
\NormalTok{knitr}\SpecialCharTok{::}\FunctionTok{kable}\NormalTok{(resum\_grp, }\AttributeTok{caption =} \StringTok{"Hauteur : moyenne (ET) par fertilisant (terreau)"}\NormalTok{)}
\end{Highlighting}
\end{Shaded}

\begin{longtable}[]{@{}lrl@{}}
\caption{Hauteur : moyenne (ET) par fertilisant
(terreau)}\tabularnewline
\toprule\noalign{}
fertilisant & n & Moyenne (ET) \\
\midrule\noalign{}
\endfirsthead
\toprule\noalign{}
fertilisant & n & Moyenne (ET) \\
\midrule\noalign{}
\endhead
\bottomrule\noalign{}
\endlastfoot
Ma & 102 & 33.76 (7.63) \\
Ca & 73 & 35.76 (7.45) \\
An & 78 & 33.53 (9.31) \\
\end{longtable}

\begin{Shaded}
\begin{Highlighting}[]
\NormalTok{knitr}\SpecialCharTok{::}\FunctionTok{kable}\NormalTok{(resum\_var, }\AttributeTok{caption =} \StringTok{"Hauteur : moyenne (ET) par variété"}\NormalTok{)}
\end{Highlighting}
\end{Shaded}

\begin{longtable}[]{@{}lrl@{}}
\caption{Hauteur : moyenne (ET) par variété}\tabularnewline
\toprule\noalign{}
variete & n & Moyenne (ET) \\
\midrule\noalign{}
\endfirsthead
\toprule\noalign{}
variete & n & Moyenne (ET) \\
\midrule\noalign{}
\endhead
\bottomrule\noalign{}
\endlastfoot
Var1 & 129 & 32.04 (8.31) \\
Var2 & 124 & 36.59 (7.35) \\
\end{longtable}

\begin{Shaded}
\begin{Highlighting}[]
\NormalTok{knitr}\SpecialCharTok{::}\FunctionTok{kable}\NormalTok{(resum\_per, }\AttributeTok{caption =} \StringTok{"Hauteur : moyenne (ET) par période"}\NormalTok{)}
\end{Highlighting}
\end{Shaded}

\begin{longtable}[]{@{}lrl@{}}
\caption{Hauteur : moyenne (ET) par période}\tabularnewline
\toprule\noalign{}
periode & n & Moyenne (ET) \\
\midrule\noalign{}
\endfirsthead
\toprule\noalign{}
periode & n & Moyenne (ET) \\
\midrule\noalign{}
\endhead
\bottomrule\noalign{}
\endlastfoot
T1 & 78 & 34.32 (9.19) \\
T2 & 57 & 35.41 (6.92) \\
T3 & 63 & 31.42 (7.89) \\
T4 & 55 & 36.27 (7.38) \\
\end{longtable}

\bookmarksetup{startatroot}

\chapter{Tableau des moyennes et écarts-types (fertilisant × variété ×
période)}\label{tableau-des-moyennes-et-uxe9carts-types-fertilisant-variuxe9tuxe9-puxe9riode}

Les statistiques descriptives de la \textbf{hauteur} sont calculées pour
chaque combinaison de \emph{fertilisant}, \emph{variété} et
\emph{période}.

\begin{Shaded}
\begin{Highlighting}[]
\CommentTok{\# {-}{-}{-} Tableau croisé fertilisant × variété × période : moyenne (ET) {-}{-}{-}}
\NormalTok{d }\SpecialCharTok{\%\textgreater{}\%}
  \FunctionTok{mutate}\NormalTok{(}\AttributeTok{fertilisant\_variete =} \FunctionTok{paste}\NormalTok{(}\FunctionTok{as.character}\NormalTok{(fertilisant), }\FunctionTok{as.character}\NormalTok{(variete), }\AttributeTok{sep =} \StringTok{" × "}\NormalTok{)) }\SpecialCharTok{\%\textgreater{}\%}
  \FunctionTok{tbl\_strata}\NormalTok{(}
    \AttributeTok{strata =}\NormalTok{ fertilisant\_variete,}
    \AttributeTok{.tbl\_fun =} \SpecialCharTok{\textasciitilde{}} \FunctionTok{tbl\_summary}\NormalTok{(.x, }\AttributeTok{by =}\NormalTok{ periode, }\AttributeTok{include =}\NormalTok{ hauteur, }\AttributeTok{statistic =} \FunctionTok{list}\NormalTok{(hauteur }\SpecialCharTok{\textasciitilde{}} \StringTok{"\{mean\} (\{sd\})"}\NormalTok{), }\AttributeTok{label =} \FunctionTok{list}\NormalTok{(hauteur }\SpecialCharTok{\textasciitilde{}} \StringTok{"Hauteur"}\NormalTok{)),}
    \AttributeTok{.header =} \StringTok{"**\{strata\}**"}\NormalTok{,}
    \AttributeTok{.combine\_with =} \StringTok{"tbl\_stack"}
\NormalTok{  ) }\SpecialCharTok{\%\textgreater{}\%}
  \FunctionTok{modify\_caption}\NormalTok{(}\StringTok{"Moyennes et écarts{-}types de la hauteur par fertilisant, variété et période"}\NormalTok{)}
\end{Highlighting}
\end{Shaded}

\begin{longtable}[]{@{}
  >{\raggedright\arraybackslash}p{(\columnwidth - 10\tabcolsep) * \real{0.1474}}
  >{\raggedright\arraybackslash}p{(\columnwidth - 10\tabcolsep) * \real{0.2000}}
  >{\centering\arraybackslash}p{(\columnwidth - 10\tabcolsep) * \real{0.1684}}
  >{\centering\arraybackslash}p{(\columnwidth - 10\tabcolsep) * \real{0.1684}}
  >{\centering\arraybackslash}p{(\columnwidth - 10\tabcolsep) * \real{0.1579}}
  >{\centering\arraybackslash}p{(\columnwidth - 10\tabcolsep) * \real{0.1579}}@{}}
\caption{Moyennes et écarts-types de la hauteur par fertilisant, variété
et période}\tabularnewline
\toprule\noalign{}
\begin{minipage}[b]{\linewidth}\raggedright
\textbf{Group}
\end{minipage} & \begin{minipage}[b]{\linewidth}\raggedright
\textbf{Characteristic}
\end{minipage} & \begin{minipage}[b]{\linewidth}\centering
\textbf{T1}, N = 11
\end{minipage} & \begin{minipage}[b]{\linewidth}\centering
\textbf{T2}, N = 10
\end{minipage} & \begin{minipage}[b]{\linewidth}\centering
\textbf{T3}, N = 9
\end{minipage} & \begin{minipage}[b]{\linewidth}\centering
\textbf{T4}, N = 5
\end{minipage} \\
\midrule\noalign{}
\endfirsthead
\toprule\noalign{}
\begin{minipage}[b]{\linewidth}\raggedright
\textbf{Group}
\end{minipage} & \begin{minipage}[b]{\linewidth}\raggedright
\textbf{Characteristic}
\end{minipage} & \begin{minipage}[b]{\linewidth}\centering
\textbf{T1}, N = 11
\end{minipage} & \begin{minipage}[b]{\linewidth}\centering
\textbf{T2}, N = 10
\end{minipage} & \begin{minipage}[b]{\linewidth}\centering
\textbf{T3}, N = 9
\end{minipage} & \begin{minipage}[b]{\linewidth}\centering
\textbf{T4}, N = 5
\end{minipage} \\
\midrule\noalign{}
\endhead
\bottomrule\noalign{}
\endlastfoot
\textbf{An × Var1} & Hauteur & 29 (10) & 34 (8) & 28 (8) & 35 (14) \\
\textbf{An × Var2} & Hauteur & 40 (8) & 36 (7) & 32 (10) & 33 (6) \\
\textbf{Ca × Var1} & Hauteur & 32 (7) & 35 (7) & 34 (4) & 34 (12) \\
\textbf{Ca × Var2} & Hauteur & 40.6 (9.1) & 36.0 (5.8) & 37.2 (6.9) &
40.3 (3.7) \\
\textbf{Ma × Var1} & Hauteur & 28 (6) & 36 (8) & 28 (8) & 36 (6) \\
\textbf{Ma × Var2} & Hauteur & 39.4 (6.5) & 33.1 (6.0) & 31.1 (5.9) &
37.7 (4.6) \\
\end{longtable}

\bookmarksetup{startatroot}

\chapter{Graphiques descriptifs}\label{graphiques-descriptifs}

\section{Distribution de la hauteur par
période}\label{distribution-de-la-hauteur-par-puxe9riode}

Boîtes à moustaches pour comparer la distribution de la hauteur à chaque
période.

\begin{Shaded}
\begin{Highlighting}[]
\FunctionTok{ggplot}\NormalTok{(d, }\FunctionTok{aes}\NormalTok{(}\AttributeTok{x =}\NormalTok{ periode, }\AttributeTok{y =}\NormalTok{ hauteur, }\AttributeTok{fill =}\NormalTok{ periode)) }\SpecialCharTok{+}
  \FunctionTok{geom\_boxplot}\NormalTok{(}\AttributeTok{alpha =} \FloatTok{0.7}\NormalTok{) }\SpecialCharTok{+}
  \FunctionTok{scale\_fill\_brewer}\NormalTok{(}\AttributeTok{palette =} \StringTok{"Pastel1"}\NormalTok{) }\SpecialCharTok{+}
  \FunctionTok{labs}\NormalTok{(}\AttributeTok{x =} \StringTok{"Période"}\NormalTok{, }\AttributeTok{y =} \StringTok{"Hauteur"}\NormalTok{) }\SpecialCharTok{+}
  \FunctionTok{theme\_minimal}\NormalTok{(}\AttributeTok{base\_size =} \DecValTok{12}\NormalTok{) }\SpecialCharTok{+}
  \FunctionTok{theme}\NormalTok{(}\AttributeTok{legend.position =} \StringTok{"none"}\NormalTok{)}
\end{Highlighting}
\end{Shaded}

\begin{figure}[H]

{\centering \includegraphics{02-analyse-descriptive_files/figure-pdf/ch2-boxplot-periode-1.pdf}

}

\caption{Distribution de la hauteur (boîtes à moustaches) par période.}

\end{figure}%

\section{Distribution de la taille par groupe
(terreau)}\label{distribution-de-la-taille-par-groupe-terreau}

\begin{Shaded}
\begin{Highlighting}[]
\FunctionTok{ggplot}\NormalTok{(d, }\FunctionTok{aes}\NormalTok{(}\AttributeTok{x =}\NormalTok{ fertilisant, }\AttributeTok{y =}\NormalTok{ hauteur, }\AttributeTok{fill =}\NormalTok{ fertilisant)) }\SpecialCharTok{+}
  \FunctionTok{geom\_boxplot}\NormalTok{(}\AttributeTok{alpha =} \FloatTok{0.7}\NormalTok{) }\SpecialCharTok{+}
  \FunctionTok{scale\_fill\_brewer}\NormalTok{(}\AttributeTok{palette =} \StringTok{"Set1"}\NormalTok{) }\SpecialCharTok{+}
  \FunctionTok{labs}\NormalTok{(}\AttributeTok{x =} \StringTok{"Fertilisant (terreau)"}\NormalTok{, }\AttributeTok{y =} \StringTok{"Hauteur"}\NormalTok{) }\SpecialCharTok{+}
  \FunctionTok{theme\_minimal}\NormalTok{(}\AttributeTok{base\_size =} \DecValTok{12}\NormalTok{) }\SpecialCharTok{+}
  \FunctionTok{theme}\NormalTok{(}\AttributeTok{legend.position =} \StringTok{"none"}\NormalTok{)}
\end{Highlighting}
\end{Shaded}

\begin{figure}[H]

{\centering \includegraphics{02-analyse-descriptive_files/figure-pdf/ch2-boxplot-groupe-1.pdf}

}

\caption{Distribution de la hauteur par fertilisant (terreau).}

\end{figure}%

\section{Graphique d'interaction (évolution selon la
période)}\label{graphique-dinteraction-uxe9volution-selon-la-puxe9riode}

Évolution des moyennes de \textbf{hauteur} en fonction de la
\textbf{période}, avec des couleurs par \textbf{fertilisant} et des
facettes par \textbf{variété}. Des courbes non parallèles suggèrent une
interaction entre période et fertilisant (ou variété).

\begin{Shaded}
\begin{Highlighting}[]
\NormalTok{d }\SpecialCharTok{\%\textgreater{}\%}
  \FunctionTok{group\_by}\NormalTok{(fertilisant, variete, periode) }\SpecialCharTok{\%\textgreater{}\%}
  \FunctionTok{summarise}\NormalTok{(}\AttributeTok{hauteur\_moy =} \FunctionTok{mean}\NormalTok{(hauteur, }\AttributeTok{na.rm =} \ConstantTok{TRUE}\NormalTok{), }\AttributeTok{.groups =} \StringTok{"drop"}\NormalTok{) }\SpecialCharTok{\%\textgreater{}\%}
  \FunctionTok{ggplot}\NormalTok{(}\FunctionTok{aes}\NormalTok{(}\AttributeTok{x =}\NormalTok{ periode, }\AttributeTok{y =}\NormalTok{ hauteur\_moy, }\AttributeTok{colour =}\NormalTok{ fertilisant, }\AttributeTok{group =}\NormalTok{ fertilisant)) }\SpecialCharTok{+}
  \FunctionTok{geom\_line}\NormalTok{(}\AttributeTok{linewidth =} \DecValTok{1}\NormalTok{) }\SpecialCharTok{+}
  \FunctionTok{geom\_point}\NormalTok{(}\AttributeTok{size =} \DecValTok{3}\NormalTok{) }\SpecialCharTok{+}
  \FunctionTok{facet\_wrap}\NormalTok{(}\SpecialCharTok{\textasciitilde{}}\NormalTok{ variete, }\AttributeTok{ncol =} \DecValTok{2}\NormalTok{) }\SpecialCharTok{+}
  \FunctionTok{scale\_color\_brewer}\NormalTok{(}\AttributeTok{palette =} \StringTok{"Set1"}\NormalTok{) }\SpecialCharTok{+}
  \FunctionTok{labs}\NormalTok{(}
    \AttributeTok{x =} \StringTok{"Période"}\NormalTok{,}
    \AttributeTok{y =} \StringTok{"Hauteur (moyenne)"}\NormalTok{,}
    \AttributeTok{colour =} \StringTok{"Fertilisant (terreau)"}
\NormalTok{  ) }\SpecialCharTok{+}
  \FunctionTok{theme\_minimal}\NormalTok{(}\AttributeTok{base\_size =} \DecValTok{12}\NormalTok{) }\SpecialCharTok{+}
  \FunctionTok{theme}\NormalTok{(}
    \AttributeTok{legend.position =} \StringTok{"bottom"}\NormalTok{,}
    \AttributeTok{strip.background =} \FunctionTok{element\_rect}\NormalTok{(}\AttributeTok{fill =} \StringTok{"grey92"}\NormalTok{, }\AttributeTok{colour =} \ConstantTok{NA}\NormalTok{)}
\NormalTok{  )}
\end{Highlighting}
\end{Shaded}

\begin{figure}[H]

{\centering \includegraphics{02-analyse-descriptive_files/figure-pdf/ch2-plot-interaction-1.pdf}

}

\caption{Évolution de la hauteur (moyenne) selon la période, par
fertilisant et par variété.}

\end{figure}%

\section{Synthèse descriptive}\label{synthuxe8se-descriptive}

Le tableau et les graphiques ci-dessus permettent de repérer
visuellement les combinaisons (fertilisant × variété × période) pour
lesquelles la hauteur moyenne est la plus élevée ou la plus faible, et
d'anticiper d'éventuelles interactions testées au chapitre 3.

\bookmarksetup{startatroot}

\chapter{Facteurs explicatifs de la variable
dépendante}\label{facteurs-explicatifs-de-la-variable-duxe9pendante}

Modèle linéaire (LM), validité et post-hoc

\hfill\break

\begin{tcolorbox}[enhanced jigsaw, opacitybacktitle=0.6, colback=white, colbacktitle=quarto-callout-important-color!10!white, leftrule=.75mm, coltitle=black, colframe=quarto-callout-important-color-frame, opacityback=0, left=2mm, rightrule=.15mm, toptitle=1mm, title=\textcolor{quarto-callout-important-color}{\faExclamation}\hspace{0.5em}{Objectif du chapitre}, arc=.35mm, bottomtitle=1mm, toprule=.15mm, breakable, bottomrule=.15mm, titlerule=0mm]

Identifier \textbf{quels facteurs} (groupe, variété, période)
\textbf{expliquent} la variable dépendante \emph{taille}, en testant les
effets principaux et les interactions (un facteur ; deux facteurs avec
ou sans interaction), puis en réalisant les comparaisons post-hoc en cas
d'effets significatifs.

\end{tcolorbox}

\bookmarksetup{startatroot}

\chapter{Objectif}\label{objectif}

Ce chapitre vise à \textbf{déterminer les facteurs qui peuvent expliquer
la variable dépendante} (la hauteur des plantes). On distingue
classiquement :

\begin{itemize}
\tightlist
\item
  \textbf{Un facteur} : on teste l'effet d'un seul facteur (ex.
  fertilisant, ou variété, ou période) sur la variable dépendante.
\item
  \textbf{Deux facteurs avec ou sans interaction} : on teste les effets
  de deux facteurs (ex. fertilisant et variété, ou fertilisant et
  période) et leur interaction. Si l'interaction est significative,
  l'effet d'un facteur dépend du niveau de l'autre ; sinon, les effets
  principaux suffisent à interpréter les résultats.
\end{itemize}

Dans notre étude, la variable dépendante est la \textbf{hauteur} ; les
facteurs sont le \textbf{fertilisant} (terreau : Ma, Ca, An), la
\textbf{variété} (Var1, Var2) et la \textbf{période} (T1--T4). Les
plantes sont toutes différentes (individus uniques), avec des mesures
répétées dans le temps. On utilise un \textbf{modèle linéaire (LM)} à
effets fixes : \textbf{fertilisant} (principal), \textbf{variété}
(contrôle) et \textbf{période} ; chaque ligne des données correspond à
une observation de hauteur.

\bookmarksetup{startatroot}

\chapter{Données et modèle linéaire
(LM)}\label{donnuxe9es-et-moduxe8le-linuxe9aire-lm}

Les mêmes données qu'au chapitre 2 sont chargées puis analysées avec un
modèle \textbf{hauteur \textasciitilde{} fertilisant × variété +
Error(id / période)} : effets fixes \emph{fertilisant} (3 terreaux),
\emph{variété} (2 variétés), \emph{période} (T1--T4) et interactions ;
effet aléatoire intercept par \emph{plante} (id). Tous les sujets sont
inclus.

\section{Explication de ce qui est fait dans le
modèle}\label{explication-de-ce-qui-est-fait-dans-le-moduxe8le}

Voici, dans l'ordre, ce que fait le chapitre en utilisant
\textbf{uniquement une ANOVA à mesures répétées avec
\texttt{afex::aov\_car}}.

\textbf{1. Import et préparation des données}

\begin{itemize}
\tightlist
\item
  Lecture du fichier Excel \texttt{donnees/donnees.xlsx}.
\item
  Variables utilisées : \textbf{groupe} (terreau Ma, Ca, An),
  \textbf{variete} (Var1, Var2), \textbf{periode} (T1--T4),
  \textbf{taille} (variable à expliquer).
\item
  Conversion de \texttt{taille} en numérique en gérant les
  \textbf{virgules décimales}.
\item
  Mise en facteur de \texttt{groupe}, \texttt{variete}, \texttt{periode}
  avec les bons niveaux.
\item
  Création d'un identifiant \textbf{technique} \texttt{id} (une
  étiquette par ligne) pour définir la structure des \textbf{mesures
  répétées} dans le modèle (mais sans entrer comme facteur explicatif).
\end{itemize}

\textbf{2. Modèle ANOVA à mesures répétées (afex::aov\_car)}

\begin{itemize}
\tightlist
\item
  Le modèle est ajusté avec :

  \begin{itemize}
  \tightlist
  \item
    \textbf{Variable dépendante (dv)} : \texttt{taille}.
  \item
    \textbf{Facteurs inter-sujets (between)} : \texttt{groupe} et
    \texttt{variete} (chaque observation appartient à une seule
    combinaison groupe × variété).
  \item
    \textbf{Facteur intra-sujet (within)} : \texttt{periode} (mêmes
    plantes suivies à T1, T2, T3, T4).
  \end{itemize}
\item
  Formule utilisée :\\
  {[} \text{taille} \sim \text{groupe} \times \text{variete} +
  \text{Error(id / periode)} {]}
\item
  \texttt{afex::aov\_car} calcule une \textbf{ANOVA de type III} avec,
  pour chaque effet (groupe, variete, groupe×variete, etc.) :

  \begin{itemize}
  \tightlist
  \item
    les \textbf{sommes de carrés},
  \item
    le \textbf{F} de Fisher,
  \item
    la \textbf{p-value},
  \item
    la \textbf{taille d'effet} ((\eta\^{}2) partiel, \texttt{pes}).
  \end{itemize}
\end{itemize}

\textbf{3. Sphéricité et corrections (Mauchly, GG, HF)}

\begin{itemize}
\tightlist
\item
  \texttt{summary()} du modèle fournit :

  \begin{itemize}
  \tightlist
  \item
    le \textbf{test de Mauchly} de sphéricité pour les effets impliquant
    \texttt{periode},
  \item
    les \textbf{corrections de Greenhouse--Geisser (GG)} et
    \textbf{Huynh--Feldt (HF)} si la sphéricité n'est pas respectée.
  \end{itemize}
\item
  L'interprétation des effets de \texttt{periode} et des interactions
  avec \texttt{periode} se base sur ces p-values corrigées.
\end{itemize}

\textbf{4. Vérifications des conditions de validité}

\begin{itemize}
\tightlist
\item
  \textbf{Normalité des résidus} : graphiques Q--Q et histogramme des
  résidus de l'ANOVA à mesures répétées.
\item
  \textbf{Homogénéité des variances (Levene)} : test de Levene sur la
  \textbf{taille moyenne par combinaison groupe × variété × id}, afin de
  vérifier l'égalité des variances entre conditions.
\end{itemize}

\textbf{5. Comparaisons post-hoc (emmeans)}

\begin{itemize}
\tightlist
\item
  \textbf{Moyennes marginales (groupe × variété)} : estimation des
  moyennes de \texttt{taille} pour chaque combinaison groupe--variété,
  avec erreurs standard et intervalles de confiance à 95 \%.
\item
  \textbf{Comparaisons deux à deux (Tukey)} :

  \begin{itemize}
  \tightlist
  \item
    Comparaisons entre toutes les combinaisons \textbf{groupe × variété}
    (effets globaux).
  \item
    Comparaisons \textbf{par période} : mêmes contrastes, mais
    séparément pour T1, T2, T3, T4, pour repérer les périodes où les
    différences sont significatives.
  \end{itemize}
\end{itemize}

\textbf{Résumé} : cette ANOVA à mesures répétées teste si le
\textbf{terreau} (\texttt{groupe}), la \textbf{variété}
(\texttt{variete}) et la \textbf{période} (\texttt{periode}) -- ainsi
que leurs interactions -- expliquent la \textbf{taille}, en respectant
la structure de mesures répétées. Les diagnostics (normalité, Levene,
sphéricité) justifient l'usage des tests F et des post-hoc de Tukey, qui
permettent ensuite de conclure sur les conditions (terreau × variété,
par période) donnant les tailles les plus élevées.

\begin{Shaded}
\begin{Highlighting}[]
\FunctionTok{library}\NormalTok{(tidyverse)}
\FunctionTok{library}\NormalTok{(emmeans)}
\FunctionTok{library}\NormalTok{(car)}
\ControlFlowTok{if}\NormalTok{ (}\SpecialCharTok{!}\FunctionTok{requireNamespace}\NormalTok{(}\StringTok{"afex"}\NormalTok{, }\AttributeTok{quietly =} \ConstantTok{TRUE}\NormalTok{)) }\FunctionTok{install.packages}\NormalTok{(}\StringTok{"afex"}\NormalTok{, }\AttributeTok{repos =} \StringTok{"https://cloud.r{-}project.org"}\NormalTok{, }\AttributeTok{quiet =} \ConstantTok{TRUE}\NormalTok{)}
\FunctionTok{library}\NormalTok{(afex)}
\ControlFlowTok{if}\NormalTok{ (}\SpecialCharTok{!}\FunctionTok{requireNamespace}\NormalTok{(}\StringTok{"gtsummary"}\NormalTok{, }\AttributeTok{quietly =} \ConstantTok{TRUE}\NormalTok{)) }\FunctionTok{install.packages}\NormalTok{(}\StringTok{"gtsummary"}\NormalTok{, }\AttributeTok{repos =} \StringTok{"https://cloud.r{-}project.org"}\NormalTok{, }\AttributeTok{quiet =} \ConstantTok{TRUE}\NormalTok{)}
\FunctionTok{library}\NormalTok{(gtsummary)}
\FunctionTok{library}\NormalTok{(knitr)}
\end{Highlighting}
\end{Shaded}

\begin{Shaded}
\begin{Highlighting}[]
\CommentTok{\# {-}{-}{-} Import : plan = mesures répétées T1–T4 pour Var1 et Var2, dans chaque fertilisant {-}{-}{-}}
\ControlFlowTok{if}\NormalTok{ (}\SpecialCharTok{!}\FunctionTok{requireNamespace}\NormalTok{(}\StringTok{"readxl"}\NormalTok{, }\AttributeTok{quietly =} \ConstantTok{TRUE}\NormalTok{)) }\FunctionTok{install.packages}\NormalTok{(}\StringTok{"readxl"}\NormalTok{, }\AttributeTok{repos =} \StringTok{"https://cloud.r{-}project.org"}\NormalTok{, }\AttributeTok{quiet =} \ConstantTok{TRUE}\NormalTok{)}
\FunctionTok{library}\NormalTok{(readxl)}
\NormalTok{d }\OtherTok{\textless{}{-}} \FunctionTok{read\_excel}\NormalTok{(}\StringTok{"donnees/donnees.xlsx"}\NormalTok{)}

\NormalTok{d}\SpecialCharTok{$}\NormalTok{hauteur }\OtherTok{\textless{}{-}} \FunctionTok{as.numeric}\NormalTok{(}\FunctionTok{gsub}\NormalTok{(}\StringTok{","}\NormalTok{, }\StringTok{"."}\NormalTok{, }\FunctionTok{as.character}\NormalTok{(d}\SpecialCharTok{$}\NormalTok{hauteur)))}
\NormalTok{d }\OtherTok{\textless{}{-}}\NormalTok{ d }\SpecialCharTok{\%\textgreater{}\%}
  \FunctionTok{mutate}\NormalTok{(}
    \AttributeTok{periode     =} \FunctionTok{factor}\NormalTok{(}\FunctionTok{as.character}\NormalTok{(periode), }\AttributeTok{levels =} \FunctionTok{c}\NormalTok{(}\StringTok{"T1"}\NormalTok{, }\StringTok{"T2"}\NormalTok{, }\StringTok{"T3"}\NormalTok{, }\StringTok{"T4"}\NormalTok{)),}
    \AttributeTok{fertilisant =} \FunctionTok{factor}\NormalTok{(fertilisant, }\AttributeTok{levels =} \FunctionTok{c}\NormalTok{(}\StringTok{"Ma"}\NormalTok{, }\StringTok{"Ca"}\NormalTok{, }\StringTok{"An"}\NormalTok{)),}
    \AttributeTok{variete     =} \FunctionTok{factor}\NormalTok{(variete)}
\NormalTok{  ) }\SpecialCharTok{\%\textgreater{}\%}
  \CommentTok{\# Création d\textquotesingle{}identifiants "sujets" artificiels par (fertilisant, variete, rang)}
  \CommentTok{\# pour simuler des mesures répétées comme dans l\textquotesingle{}exemple DABIRE}
  \FunctionTok{group\_by}\NormalTok{(fertilisant, variete, periode) }\SpecialCharTok{\%\textgreater{}\%}
  \FunctionTok{arrange}\NormalTok{(fertilisant, variete, periode, }\AttributeTok{.by\_group =} \ConstantTok{TRUE}\NormalTok{) }\SpecialCharTok{\%\textgreater{}\%}
  \FunctionTok{mutate}\NormalTok{(}\AttributeTok{rang =} \FunctionTok{row\_number}\NormalTok{()) }\SpecialCharTok{\%\textgreater{}\%}
  \FunctionTok{ungroup}\NormalTok{() }\SpecialCharTok{\%\textgreater{}\%}
  \FunctionTok{mutate}\NormalTok{(}
    \AttributeTok{id =} \FunctionTok{interaction}\NormalTok{(fertilisant, variete, rang, }\AttributeTok{drop =} \ConstantTok{TRUE}\NormalTok{),}
    \AttributeTok{id =} \FunctionTok{factor}\NormalTok{(id)}
\NormalTok{  )}
\end{Highlighting}
\end{Shaded}

\section{Description du modèle
utilisé}\label{description-du-moduxe8le-utilisuxe9}

Le modèle utilisé est une \textbf{ANOVA à mesures répétées} implémentée
avec \texttt{afex::aov\_car}. Chaque ligne du tableau correspond à
\textbf{une observation de hauteur pour une combinaison (fertilisant,
variété, période)}.

\begin{longtable}[]{@{}
  >{\raggedright\arraybackslash}p{(\columnwidth - 2\tabcolsep) * \real{0.5000}}
  >{\raggedright\arraybackslash}p{(\columnwidth - 2\tabcolsep) * \real{0.5000}}@{}}
\toprule\noalign{}
\begin{minipage}[b]{\linewidth}\raggedright
Élément
\end{minipage} & \begin{minipage}[b]{\linewidth}\raggedright
Description
\end{minipage} \\
\midrule\noalign{}
\endhead
\bottomrule\noalign{}
\endlastfoot
\textbf{Variable dépendante} & \texttt{hauteur} (hauteur des plantes,
quantitative) \\
\textbf{Facteurs inter-sujets} & \texttt{fertilisant} (3 terreaux : Ma,
Ca, An), \texttt{variete} (2 variétés) \\
\textbf{Facteur intra-sujet} & \texttt{periode} (T1--T4, mesures
répétées) \\
\textbf{Identifiant (id)} & Étiquette technique construite par
combinaison \texttt{(fertilisant,\ variete,\ rang)} pour regrouper
artificiellement les mesures répétées dans
\texttt{Error(id\ /\ periode)} \\
\textbf{Formule} &
\texttt{hauteur\ \textasciitilde{}\ fertilisant\ *\ variete\ +\ Error(id\ /\ periode)} \\
\textbf{Options} & Type III ; corrections \textbf{Greenhouse--Geisser} /
\textbf{Huynh--Feldt} pour les termes impliquant \texttt{periode} ;
\textbf{η² partiel (pes)} comme taille d'effet \\
\end{longtable}

\begin{Shaded}
\begin{Highlighting}[]
\CommentTok{\# {-}{-}{-} ANOVA à mesures répétées avec afex::aov\_car {-}{-}{-}{-}{-}{-}{-}{-}{-}{-}{-}{-}{-}{-}{-}{-}{-}{-}{-}{-}{-}{-}{-}{-}{-}}

\CommentTok{\# On fixe le type de sommes de carrés à III (classique en présence d\textquotesingle{}interactions)}
\NormalTok{afex}\SpecialCharTok{::}\FunctionTok{afex\_options}\NormalTok{(}\AttributeTok{type =} \DecValTok{3}\NormalTok{)}

\CommentTok{\# Ajustement du modèle :}
\CommentTok{\# {-} hauteur : variable dépendante}
\CommentTok{\# {-} fertilisant * variete : effets fixes inter{-}sujets et leur interaction}
\CommentTok{\# {-} Error(id / periode) : structure de mesures répétées (effet de periode imbriqué dans id)}
\CommentTok{\# {-} na.rm = TRUE : les observations manquantes (périodes non observées pour certains individus)}
\CommentTok{\#   sont retirées de l\textquotesingle{}analyse, comme suggéré par le message d\textquotesingle{}erreur d\textquotesingle{}afex.}
\NormalTok{anova\_rm }\OtherTok{\textless{}{-}}\NormalTok{ afex}\SpecialCharTok{::}\FunctionTok{aov\_car}\NormalTok{(}
\NormalTok{  hauteur }\SpecialCharTok{\textasciitilde{}}\NormalTok{ fertilisant }\SpecialCharTok{*}\NormalTok{ variete }\SpecialCharTok{+} \FunctionTok{Error}\NormalTok{(id }\SpecialCharTok{/}\NormalTok{ periode),}
  \AttributeTok{data  =}\NormalTok{ d,}
  \AttributeTok{na.rm =} \ConstantTok{TRUE}
\NormalTok{)}

\CommentTok{\# Tableau ANOVA "propre" (nice) :}
\CommentTok{\# {-} fournit F, p{-}value, et η² partiel (pes) pour chaque effet}
\NormalTok{tab\_rm }\OtherTok{\textless{}{-}} \FunctionTok{nice}\NormalTok{(anova\_rm, }\AttributeTok{es =} \StringTok{"pes"}\NormalTok{)}
\NormalTok{knitr}\SpecialCharTok{::}\FunctionTok{kable}\NormalTok{(}
\NormalTok{  tab\_rm,}
  \AttributeTok{caption =} \StringTok{"ANOVA à mesures répétées (afex::aov\_car) : effets de fertilisant, variété, période et interactions, avec η² partiel."}
\NormalTok{)}
\end{Highlighting}
\end{Shaded}

\begin{longtable}[]{@{}
  >{\raggedright\arraybackslash}p{(\columnwidth - 10\tabcolsep) * \real{0.4118}}
  >{\raggedright\arraybackslash}p{(\columnwidth - 10\tabcolsep) * \real{0.1765}}
  >{\raggedright\arraybackslash}p{(\columnwidth - 10\tabcolsep) * \real{0.0882}}
  >{\raggedright\arraybackslash}p{(\columnwidth - 10\tabcolsep) * \real{0.1324}}
  >{\raggedright\arraybackslash}p{(\columnwidth - 10\tabcolsep) * \real{0.0735}}
  >{\raggedright\arraybackslash}p{(\columnwidth - 10\tabcolsep) * \real{0.1176}}@{}}
\caption{ANOVA à mesures répétées (afex::aov\_car) : effets de
fertilisant, variété, période et interactions, avec η²
partiel.}\tabularnewline
\toprule\noalign{}
\begin{minipage}[b]{\linewidth}\raggedright
Effect
\end{minipage} & \begin{minipage}[b]{\linewidth}\raggedright
df
\end{minipage} & \begin{minipage}[b]{\linewidth}\raggedright
MSE
\end{minipage} & \begin{minipage}[b]{\linewidth}\raggedright
F
\end{minipage} & \begin{minipage}[b]{\linewidth}\raggedright
pes
\end{minipage} & \begin{minipage}[b]{\linewidth}\raggedright
p.value
\end{minipage} \\
\midrule\noalign{}
\endfirsthead
\toprule\noalign{}
\begin{minipage}[b]{\linewidth}\raggedright
Effect
\end{minipage} & \begin{minipage}[b]{\linewidth}\raggedright
df
\end{minipage} & \begin{minipage}[b]{\linewidth}\raggedright
MSE
\end{minipage} & \begin{minipage}[b]{\linewidth}\raggedright
F
\end{minipage} & \begin{minipage}[b]{\linewidth}\raggedright
pes
\end{minipage} & \begin{minipage}[b]{\linewidth}\raggedright
p.value
\end{minipage} \\
\midrule\noalign{}
\endhead
\bottomrule\noalign{}
\endlastfoot
fertilisant & 2, 35 & 50.78 & 4.53 * & .205 & .018 \\
variete & 1, 35 & 50.78 & 10.51 ** & .231 & .003 \\
fertilisant:variete & 2, 35 & 50.78 & 0.09 & .005 & .916 \\
periode & 2.59, 90.49 & 73.59 & 5.29 ** & .131 & .003 \\
fertilisant:periode & 5.17, 90.49 & 73.59 & 0.49 & .027 & .792 \\
variete:periode & 2.59, 90.49 & 73.59 & 4.77 ** & .120 & .006 \\
fertilisant:variete:periode & 5.17, 90.49 & 73.59 & 1.05 & .056 &
.396 \\
\end{longtable}

\begin{Shaded}
\begin{Highlighting}[]
\CommentTok{\# Résumé complet du modèle :}
\CommentTok{\# {-} inclut le test de Mauchly (sphéricité)}
\CommentTok{\# {-} et les corrections GG / HF si nécessaire}
\FunctionTok{summary}\NormalTok{(anova\_rm)}
\end{Highlighting}
\end{Shaded}

\begin{verbatim}

Univariate Type III Repeated-Measures ANOVA Assuming Sphericity

                            Sum Sq num Df Error SS den Df   F value    Pr(>F)
(Intercept)                 168766      1   1777.2     35 3323.6960 < 2.2e-16
fertilisant                    460      2   1777.2     35    4.5250  0.017870
variete                        534      1   1777.2     35   10.5127  0.002605
fertilisant:variete              9      2   1777.2     35    0.0878  0.916122
periode                       1006      3   6659.0    105    5.2876  0.001956
fertilisant:periode            185      6   6659.0    105    0.4853  0.818020
variete:periode                908      3   6659.0    105    4.7733  0.003694
fertilisant:variete:periode    399      6   6659.0    105    1.0478  0.398892
                               
(Intercept)                 ***
fertilisant                 *  
variete                     ** 
fertilisant:variete            
periode                     ** 
fertilisant:periode            
variete:periode             ** 
fertilisant:variete:periode    
---
Signif. codes:  0 '***' 0.001 '**' 0.01 '*' 0.05 '.' 0.1 ' ' 1


Mauchly Tests for Sphericity

                            Test statistic p-value
periode                            0.80217 0.19052
fertilisant:periode                0.80217 0.19052
variete:periode                    0.80217 0.19052
fertilisant:variete:periode        0.80217 0.19052


Greenhouse-Geisser and Huynh-Feldt Corrections
 for Departure from Sphericity

                             GG eps Pr(>F[GG])   
periode                     0.86184   0.003361 **
fertilisant:periode         0.86184   0.792291   
variete:periode             0.86184   0.005876 **
fertilisant:variete:periode 0.86184   0.395740   
---
Signif. codes:  0 '***' 0.001 '**' 0.01 '*' 0.05 '.' 0.1 ' ' 1

                               HF eps  Pr(>F[HF])
periode                     0.9365994 0.002506562
fertilisant:periode         0.9365994 0.806780783
variete:periode             0.9365994 0.004568788
fertilisant:variete:periode 0.9365994 0.397572844
\end{verbatim}

En résumé, la hauteur dépend significativement du \textbf{fertilisant}
et de la \textbf{variété}, ainsi que de la \textbf{période} de mesure,
avec une interaction significative entre \textbf{variété et période} ;
les interactions impliquant le fertilisant ne sont pas significatives.

Les tests de \textbf{Mauchly} ne sont pas significatifs, ce qui indique
que l'hypothèse de sphéricité est raisonnable ; les corrections de
Greenhouse--Geisser et Huynh--Feldt confirment la significativité de
l'effet de la \textbf{période} et de l'interaction \textbf{variété ×
période}, tandis que les termes avec le fertilisant restent non
significatifs.

\bookmarksetup{startatroot}

\chapter{Vérification des conditions de
validité}\label{vuxe9rification-des-conditions-de-validituxe9}

\section{Normalité des résidus}\label{normalituxe9-des-ruxe9sidus}

\begin{Shaded}
\begin{Highlighting}[]
\CommentTok{\# On extrait les résidus du modèle sous{-}jacent (lm) contenu dans l\textquotesingle{}objet afex}
\NormalTok{res }\OtherTok{\textless{}{-}} \FunctionTok{residuals}\NormalTok{(anova\_rm}\SpecialCharTok{$}\NormalTok{lm)}

\CommentTok{\# Affichage côte à côte :}
\CommentTok{\# {-} Q{-}Q plot : vérification de l\textquotesingle{}alignement des résidus sur la droite théorique}
\CommentTok{\# {-} Histogramme : forme globale (approximation gaussienne attendue)}
\FunctionTok{par}\NormalTok{(}\AttributeTok{mfrow =} \FunctionTok{c}\NormalTok{(}\DecValTok{1}\NormalTok{, }\DecValTok{2}\NormalTok{))}
\FunctionTok{qqnorm}\NormalTok{(res, }\AttributeTok{main =} \StringTok{"Q{-}Q des résidus"}\NormalTok{); }\FunctionTok{qqline}\NormalTok{(res)}
\FunctionTok{hist}\NormalTok{(res, }\AttributeTok{breaks =} \DecValTok{15}\NormalTok{, }\AttributeTok{main =} \StringTok{"Distribution des résidus"}\NormalTok{, }\AttributeTok{xlab =} \StringTok{"Résidus"}\NormalTok{)}
\end{Highlighting}
\end{Shaded}

\begin{figure}[H]

{\centering \includegraphics{03-facteurs-explicatifs_files/figure-pdf/normalite-1.pdf}

}

\caption{Graphique Q-Q et histogramme des résidus de l'ANOVA à mesures
répétées.}

\end{figure}%

\begin{Shaded}
\begin{Highlighting}[]
\FunctionTok{par}\NormalTok{(}\AttributeTok{mfrow =} \FunctionTok{c}\NormalTok{(}\DecValTok{1}\NormalTok{, }\DecValTok{1}\NormalTok{))}

\CommentTok{\# Test formel de normalité (Shapiro{-}Wilk)}
\NormalTok{shapiro\_res }\OtherTok{\textless{}{-}} \FunctionTok{shapiro.test}\NormalTok{(res)}
\NormalTok{shapiro\_res}
\end{Highlighting}
\end{Shaded}

\begin{verbatim}

    Shapiro-Wilk normality test

data:  res
W = 0.96094, p-value = 0.0001458
\end{verbatim}

Les graphes de résidus indiquent une distribution globalement symétrique
et proche d'une droite normale, mais le test de Shapiro--Wilk est
significatif (p-value très petite), ce qui met en évidence une déviation
statistique de la normalité ; compte tenu de la taille d'échantillon et
de l'homogénéité des variances, l'ANOVA reste toutefois utilisable avec
prudence.

\section{Homogénéité des variances
(Levene)}\label{homoguxe9nuxe9ituxe9-des-variances-levene}

Le test de Levene est réalisé sur la \textbf{hauteur moyenne par
combinaison (id, fertilisant, variete)}, ce qui revient à comparer les
variances entre conditions expérimentales (fertilisant × variété),
indépendamment de la période.

\begin{Shaded}
\begin{Highlighting}[]
\CommentTok{\# Agrégation : moyenne de hauteur par id, fertilisant, variété}
\NormalTok{d\_mean }\OtherTok{\textless{}{-}}\NormalTok{ d }\SpecialCharTok{\%\textgreater{}\%}
  \FunctionTok{group\_by}\NormalTok{(id, fertilisant, variete) }\SpecialCharTok{\%\textgreater{}\%}
  \FunctionTok{summarise}\NormalTok{(}\AttributeTok{hauteur\_moy =} \FunctionTok{mean}\NormalTok{(hauteur, }\AttributeTok{na.rm =} \ConstantTok{TRUE}\NormalTok{), }\AttributeTok{.groups =} \StringTok{"drop"}\NormalTok{)}

\CommentTok{\# Test de Levene sur les variances de hauteur moyenne entre combinaisons fertilisant × variété}
\NormalTok{lev }\OtherTok{\textless{}{-}}\NormalTok{ car}\SpecialCharTok{::}\FunctionTok{leveneTest}\NormalTok{(hauteur\_moy }\SpecialCharTok{\textasciitilde{}}\NormalTok{ fertilisant }\SpecialCharTok{*}\NormalTok{ variete, }\AttributeTok{data =}\NormalTok{ d\_mean)}

\CommentTok{\# Mise en forme d\textquotesingle{}un tableau lisible}
\NormalTok{lev\_df }\OtherTok{\textless{}{-}} \FunctionTok{as.data.frame}\NormalTok{(lev)}
\NormalTok{lev\_df }\OtherTok{\textless{}{-}} \FunctionTok{cbind}\NormalTok{(}\AttributeTok{Source =} \FunctionTok{rownames}\NormalTok{(lev\_df), lev\_df)}
\FunctionTok{rownames}\NormalTok{(lev\_df) }\OtherTok{\textless{}{-}} \ConstantTok{NULL}
\FunctionTok{names}\NormalTok{(lev\_df) }\OtherTok{\textless{}{-}} \FunctionTok{c}\NormalTok{(}\StringTok{"Source"}\NormalTok{, }\StringTok{"Ddl"}\NormalTok{, }\StringTok{"F"}\NormalTok{, }\StringTok{"p{-}value"}\NormalTok{)}
\NormalTok{lev\_df[[}\StringTok{"p{-}value"}\NormalTok{]] }\OtherTok{\textless{}{-}} \FunctionTok{ifelse}\NormalTok{(}\FunctionTok{is.na}\NormalTok{(lev\_df[[}\StringTok{"p{-}value"}\NormalTok{]]), }\StringTok{"—"}\NormalTok{, }\FunctionTok{format.pval}\NormalTok{(lev\_df[[}\StringTok{"p{-}value"}\NormalTok{]], }\AttributeTok{digits =} \DecValTok{3}\NormalTok{))}
\NormalTok{lev\_df[[}\StringTok{"F"}\NormalTok{]] }\OtherTok{\textless{}{-}} \FunctionTok{ifelse}\NormalTok{(}\FunctionTok{is.na}\NormalTok{(lev\_df[[}\StringTok{"F"}\NormalTok{]]), }\StringTok{"—"}\NormalTok{, }\FunctionTok{format}\NormalTok{(}\FunctionTok{round}\NormalTok{(lev\_df[[}\StringTok{"F"}\NormalTok{]], }\DecValTok{4}\NormalTok{), }\AttributeTok{nsmall =} \DecValTok{4}\NormalTok{))}
\NormalTok{lev\_df[}\FunctionTok{is.na}\NormalTok{(lev\_df)] }\OtherTok{\textless{}{-}} \StringTok{"—"}

\NormalTok{knitr}\SpecialCharTok{::}\FunctionTok{kable}\NormalTok{(}
\NormalTok{  lev\_df,}
  \AttributeTok{align =} \FunctionTok{c}\NormalTok{(}\StringTok{"l"}\NormalTok{, }\StringTok{"r"}\NormalTok{, }\StringTok{"r"}\NormalTok{, }\StringTok{"r"}\NormalTok{),}
  \AttributeTok{caption =} \StringTok{"Test de Levene sur la hauteur moyenne (homogénéité des variances entre groupes de fertilisant et de variété)."}
\NormalTok{)}
\end{Highlighting}
\end{Shaded}

\begin{longtable}[]{@{}lrrr@{}}
\caption{Test de Levene sur la hauteur moyenne (homogénéité des
variances entre groupes de fertilisant et de variété).}\tabularnewline
\toprule\noalign{}
Source & Ddl & F & p-value \\
\midrule\noalign{}
\endfirsthead
\toprule\noalign{}
Source & Ddl & F & p-value \\
\midrule\noalign{}
\endhead
\bottomrule\noalign{}
\endlastfoot
group & 5 & 0.8640 & 0.509 \\
& 75 & --- & --- \\
\end{longtable}

Le test de Levene n'étant pas significatif, on peut considérer que les
variances de hauteur sont similaires entre les combinaisons de
fertilisant et de variété.

\bookmarksetup{startatroot}

\chapter{Comparaisons post-hoc}\label{comparaisons-post-hoc}

\bookmarksetup{startatroot}

\chapter{Comparaisons post-hoc}\label{comparaisons-post-hoc-1}

Lorsque des effets ou interactions sont significatifs, les comparaisons
deux à deux (Tukey) permettent d'identifier les combinaisons
\textbf{groupe × variété} et les \textbf{périodes} dont les moyennes
diffèrent significativement.

\section{Moyennes marginales (fertilisant ×
variété)}\label{moyennes-marginales-fertilisant-variuxe9tuxe9}

\begin{Shaded}
\begin{Highlighting}[]
\CommentTok{\# Moyennes marginales (effet global de fertilisant × variété, toutes périodes confondues)}
\NormalTok{emm\_gv }\OtherTok{\textless{}{-}} \FunctionTok{emmeans}\NormalTok{(anova\_rm, }\SpecialCharTok{\textasciitilde{}}\NormalTok{ fertilisant }\SpecialCharTok{*}\NormalTok{ variete)}

\CommentTok{\# Extraction sous forme de data.frame et arrondi pour la présentation}
\NormalTok{emm\_gv\_df }\OtherTok{\textless{}{-}} \FunctionTok{summary}\NormalTok{(emm\_gv) }\SpecialCharTok{\%\textgreater{}\%} \FunctionTok{as.data.frame}\NormalTok{()}
\NormalTok{emm\_gv\_df }\OtherTok{\textless{}{-}}\NormalTok{ emm\_gv\_df }\SpecialCharTok{\%\textgreater{}\%}
  \FunctionTok{mutate}\NormalTok{(}
    \AttributeTok{emmean   =} \FunctionTok{round}\NormalTok{(emmean, }\DecValTok{2}\NormalTok{),}
    \AttributeTok{SE       =} \FunctionTok{round}\NormalTok{(SE, }\DecValTok{3}\NormalTok{),}
    \AttributeTok{lower.CL =} \FunctionTok{round}\NormalTok{(lower.CL, }\DecValTok{2}\NormalTok{),}
    \AttributeTok{upper.CL =} \FunctionTok{round}\NormalTok{(upper.CL, }\DecValTok{2}\NormalTok{)}
\NormalTok{  )}

\FunctionTok{names}\NormalTok{(emm\_gv\_df) }\OtherTok{\textless{}{-}} \FunctionTok{c}\NormalTok{(}\StringTok{"Fertilisant"}\NormalTok{, }\StringTok{"Variété"}\NormalTok{, }\StringTok{"Moyenne"}\NormalTok{, }\StringTok{"ESM"}\NormalTok{, }\StringTok{"Ddl"}\NormalTok{, }\StringTok{"IC inf."}\NormalTok{, }\StringTok{"IC sup."}\NormalTok{)}

\NormalTok{knitr}\SpecialCharTok{::}\FunctionTok{kable}\NormalTok{(}
\NormalTok{  emm\_gv\_df[, }\FunctionTok{c}\NormalTok{(}\StringTok{"Fertilisant"}\NormalTok{, }\StringTok{"Variété"}\NormalTok{, }\StringTok{"Moyenne"}\NormalTok{, }\StringTok{"ESM"}\NormalTok{, }\StringTok{"IC inf."}\NormalTok{, }\StringTok{"IC sup."}\NormalTok{)],}
  \AttributeTok{align =} \FunctionTok{c}\NormalTok{(}\StringTok{"l"}\NormalTok{, }\StringTok{"l"}\NormalTok{, }\StringTok{"r"}\NormalTok{, }\StringTok{"r"}\NormalTok{, }\StringTok{"r"}\NormalTok{, }\StringTok{"r"}\NormalTok{),}
  \AttributeTok{caption =} \StringTok{"Moyennes marginales (et IC 95 \%) pour fertilisant × variété (toutes périodes confondues)."}
\NormalTok{)}
\end{Highlighting}
\end{Shaded}

\begin{longtable}[]{@{}llrrrr@{}}
\caption{Moyennes marginales (et IC 95 \%) pour fertilisant × variété
(toutes périodes confondues).}\tabularnewline
\toprule\noalign{}
Fertilisant & Variété & Moyenne & ESM & IC inf. & IC sup. \\
\midrule\noalign{}
\endfirsthead
\toprule\noalign{}
Fertilisant & Variété & Moyenne & ESM & IC inf. & IC sup. \\
\midrule\noalign{}
\endhead
\bottomrule\noalign{}
\endlastfoot
Ma & Var1 & 31.58 & 1.029 & 29.49 & 33.67 \\
Ca & Var1 & 34.61 & 1.347 & 31.88 & 37.34 \\
An & Var1 & 30.04 & 1.593 & 26.81 & 33.27 \\
Ma & Var2 & 36.03 & 1.781 & 32.41 & 39.64 \\
Ca & Var2 & 37.86 & 1.455 & 34.91 & 40.81 \\
An & Var2 & 33.82 & 1.347 & 31.08 & 36.55 \\
\end{longtable}

Ces moyennes montrent que, en moyenne sur toutes les périodes, la
variété~2 produit des plantes plus hautes que la variété~1 pour un même
fertilisant, avec des valeurs maximales observées pour les combinaisons
\texttt{Ca–Var2} et \texttt{Ma–Var2}.

\section{Comparaisons deux à deux : fertilisant × variété
(Tukey)}\label{comparaisons-deux-uxe0-deux-fertilisant-variuxe9tuxe9-tukey}

\begin{Shaded}
\begin{Highlighting}[]
\CommentTok{\# Comparaisons toutes paires de combinaisons fertilisant × variété (corrigées de Tukey)}
\NormalTok{ph\_gv }\OtherTok{\textless{}{-}} \FunctionTok{pairs}\NormalTok{(emm\_gv, }\AttributeTok{adjust =} \StringTok{"tukey"}\NormalTok{) }\SpecialCharTok{\%\textgreater{}\%} \FunctionTok{as.data.frame}\NormalTok{()}
\NormalTok{pval }\OtherTok{\textless{}{-}}\NormalTok{ ph\_gv}\SpecialCharTok{$}\NormalTok{p.value}

\NormalTok{ph\_gv }\OtherTok{\textless{}{-}}\NormalTok{ ph\_gv }\SpecialCharTok{\%\textgreater{}\%}
  \FunctionTok{mutate}\NormalTok{(}
    \AttributeTok{estimate =} \FunctionTok{round}\NormalTok{(estimate, }\DecValTok{3}\NormalTok{),}
    \AttributeTok{SE       =} \FunctionTok{round}\NormalTok{(SE, }\DecValTok{3}\NormalTok{),}
    \AttributeTok{t.ratio  =} \FunctionTok{round}\NormalTok{(t.ratio, }\DecValTok{2}\NormalTok{),}
    \AttributeTok{signif   =} \FunctionTok{case\_when}\NormalTok{(}
\NormalTok{      pval }\SpecialCharTok{\textless{}} \FloatTok{0.001} \SpecialCharTok{\textasciitilde{}} \StringTok{"***"}\NormalTok{,}
\NormalTok{      pval }\SpecialCharTok{\textless{}} \FloatTok{0.01}  \SpecialCharTok{\textasciitilde{}} \StringTok{"**"}\NormalTok{,}
\NormalTok{      pval }\SpecialCharTok{\textless{}} \FloatTok{0.05}  \SpecialCharTok{\textasciitilde{}} \StringTok{"*"}\NormalTok{,}
      \ConstantTok{TRUE}         \SpecialCharTok{\textasciitilde{}} \StringTok{""}
\NormalTok{    ),}
    \AttributeTok{p.value =} \FunctionTok{format.pval}\NormalTok{(p.value, }\AttributeTok{digits =} \DecValTok{3}\NormalTok{, }\AttributeTok{eps =} \FloatTok{0.001}\NormalTok{)}
\NormalTok{  )}

\FunctionTok{names}\NormalTok{(ph\_gv) }\OtherTok{\textless{}{-}} \FunctionTok{c}\NormalTok{(}\StringTok{"Contraste"}\NormalTok{, }\StringTok{"Différence"}\NormalTok{, }\StringTok{"ESM"}\NormalTok{, }\StringTok{"Ddl"}\NormalTok{, }\StringTok{"t"}\NormalTok{, }\StringTok{"p{-}value"}\NormalTok{, }\StringTok{"Signif."}\NormalTok{)}

\NormalTok{knitr}\SpecialCharTok{::}\FunctionTok{kable}\NormalTok{(}
\NormalTok{  ph\_gv,}
  \AttributeTok{align =} \FunctionTok{c}\NormalTok{(}\StringTok{"l"}\NormalTok{, }\StringTok{"r"}\NormalTok{, }\StringTok{"r"}\NormalTok{, }\StringTok{"r"}\NormalTok{, }\StringTok{"r"}\NormalTok{, }\StringTok{"r"}\NormalTok{, }\StringTok{"c"}\NormalTok{),}
  \AttributeTok{caption =} \StringTok{"Comparaisons post{-}hoc Tukey : fertilisant × variété (toutes périodes confondues). * p \textless{} 0,05 ; ** p \textless{} 0,01 ; *** p \textless{} 0,001."}
\NormalTok{)}
\end{Highlighting}
\end{Shaded}

\begin{longtable}[]{@{}lrrrrrc@{}}
\caption{Comparaisons post-hoc Tukey : fertilisant × variété (toutes
périodes confondues). * p \textless{} 0,05 ; ** p \textless{} 0,01 ; ***
p \textless{} 0,001.}\tabularnewline
\toprule\noalign{}
Contraste & Différence & ESM & Ddl & t & p-value & Signif. \\
\midrule\noalign{}
\endfirsthead
\toprule\noalign{}
Contraste & Différence & ESM & Ddl & t & p-value & Signif. \\
\midrule\noalign{}
\endhead
\bottomrule\noalign{}
\endlastfoot
Ma Var1 - Ca Var1 & -3.029 & 1.694 & 35 & -1.79 & 0.4863 & \\
Ma Var1 - An Var1 & 1.541 & 1.896 & 35 & 0.81 & 0.9633 & \\
Ma Var1 - Ma Var2 & -4.444 & 2.057 & 35 & -2.16 & 0.2816 & \\
Ma Var1 - Ca Var2 & -6.277 & 1.781 & 35 & -3.52 & 0.0141 & * \\
Ma Var1 - An Var2 & -2.237 & 1.694 & 35 & -1.32 & 0.7722 & \\
Ca Var1 - An Var1 & 4.571 & 2.086 & 35 & 2.19 & 0.2675 & \\
Ca Var1 - Ma Var2 & -1.414 & 2.233 & 35 & -0.63 & 0.9877 & \\
Ca Var1 - Ca Var2 & -3.248 & 1.982 & 35 & -1.64 & 0.5798 & \\
Ca Var1 - An Var2 & 0.793 & 1.904 & 35 & 0.42 & 0.9983 & \\
An Var1 - Ma Var2 & -5.985 & 2.390 & 35 & -2.50 & 0.1505 & \\
An Var1 - Ca Var2 & -7.818 & 2.157 & 35 & -3.62 & 0.0109 & * \\
An Var1 - An Var2 & -3.778 & 2.086 & 35 & -1.81 & 0.4722 & \\
Ma Var2 - Ca Var2 & -1.833 & 2.300 & 35 & -0.80 & 0.9661 & \\
Ma Var2 - An Var2 & 2.207 & 2.233 & 35 & 0.99 & 0.9186 & \\
Ca Var2 - An Var2 & 4.040 & 1.982 & 35 & 2.04 & 0.3422 & \\
\end{longtable}

Après correction de Tukey, seules les combinaisons \texttt{Ca–Var2} vs
\texttt{Ma–Var1} et \texttt{Ca–Var2} vs \texttt{An–Var1} présentent des
différences significatives de hauteur, ce qui confirme un avantage
marqué de \texttt{Ca–Var2} par rapport à ces deux couples.

\section{Comparaisons deux à deux par période
(Tukey)}\label{comparaisons-deux-uxe0-deux-par-puxe9riode-tukey}

\begin{Shaded}
\begin{Highlighting}[]
\CommentTok{\# Effets de fertilisant × variété examinés séparément pour chaque période (T1, T2, T3, T4)}
\NormalTok{emm\_per }\OtherTok{\textless{}{-}} \FunctionTok{emmeans}\NormalTok{(anova\_rm, }\SpecialCharTok{\textasciitilde{}}\NormalTok{ fertilisant }\SpecialCharTok{*}\NormalTok{ variete }\SpecialCharTok{|}\NormalTok{ periode)}
\NormalTok{ph\_per }\OtherTok{\textless{}{-}} \FunctionTok{pairs}\NormalTok{(emm\_per, }\AttributeTok{adjust =} \StringTok{"tukey"}\NormalTok{) }\SpecialCharTok{\%\textgreater{}\%} \FunctionTok{as.data.frame}\NormalTok{()}
\NormalTok{pval\_per }\OtherTok{\textless{}{-}}\NormalTok{ ph\_per}\SpecialCharTok{$}\NormalTok{p.value}

\NormalTok{ph\_per }\OtherTok{\textless{}{-}}\NormalTok{ ph\_per }\SpecialCharTok{\%\textgreater{}\%}
  \FunctionTok{mutate}\NormalTok{(}
    \AttributeTok{estimate =} \FunctionTok{round}\NormalTok{(estimate, }\DecValTok{3}\NormalTok{),}
    \AttributeTok{SE       =} \FunctionTok{round}\NormalTok{(SE, }\DecValTok{3}\NormalTok{),}
    \AttributeTok{t.ratio  =} \FunctionTok{round}\NormalTok{(t.ratio, }\DecValTok{2}\NormalTok{),}
    \AttributeTok{signif   =} \FunctionTok{case\_when}\NormalTok{(}
\NormalTok{      pval\_per }\SpecialCharTok{\textless{}} \FloatTok{0.001} \SpecialCharTok{\textasciitilde{}} \StringTok{"***"}\NormalTok{,}
\NormalTok{      pval\_per }\SpecialCharTok{\textless{}} \FloatTok{0.01}  \SpecialCharTok{\textasciitilde{}} \StringTok{"**"}\NormalTok{,}
\NormalTok{      pval\_per }\SpecialCharTok{\textless{}} \FloatTok{0.05}  \SpecialCharTok{\textasciitilde{}} \StringTok{"*"}\NormalTok{,}
      \ConstantTok{TRUE}             \SpecialCharTok{\textasciitilde{}} \StringTok{""}
\NormalTok{    ),}
    \AttributeTok{p.value =} \FunctionTok{format.pval}\NormalTok{(p.value, }\AttributeTok{digits =} \DecValTok{3}\NormalTok{, }\AttributeTok{eps =} \FloatTok{0.001}\NormalTok{)}
\NormalTok{  )}

\FunctionTok{names}\NormalTok{(ph\_per) }\OtherTok{\textless{}{-}} \FunctionTok{c}\NormalTok{(}\StringTok{"Contraste"}\NormalTok{, }\StringTok{"Différence"}\NormalTok{, }\StringTok{"ESM"}\NormalTok{, }\StringTok{"Ddl"}\NormalTok{, }\StringTok{"t"}\NormalTok{, }\StringTok{"p{-}value"}\NormalTok{, }\StringTok{"Signif."}\NormalTok{, }\StringTok{"Période"}\NormalTok{)}

\NormalTok{knitr}\SpecialCharTok{::}\FunctionTok{kable}\NormalTok{(}
\NormalTok{  ph\_per[, }\FunctionTok{c}\NormalTok{(}\StringTok{"Période"}\NormalTok{, }\StringTok{"Contraste"}\NormalTok{, }\StringTok{"Différence"}\NormalTok{, }\StringTok{"ESM"}\NormalTok{, }\StringTok{"t"}\NormalTok{, }\StringTok{"p{-}value"}\NormalTok{, }\StringTok{"Signif."}\NormalTok{)],}
  \AttributeTok{align =} \FunctionTok{c}\NormalTok{(}\StringTok{"l"}\NormalTok{, }\StringTok{"l"}\NormalTok{, }\StringTok{"r"}\NormalTok{, }\StringTok{"r"}\NormalTok{, }\StringTok{"r"}\NormalTok{, }\StringTok{"r"}\NormalTok{, }\StringTok{"c"}\NormalTok{),}
  \AttributeTok{caption =} \StringTok{"Comparaisons post{-}hoc Tukey par période. * p \textless{} 0,05 ; ** p \textless{} 0,01 ; *** p \textless{} 0,001."}
\NormalTok{)}
\end{Highlighting}
\end{Shaded}

\begin{longtable}[]{@{}
  >{\raggedright\arraybackslash}p{(\columnwidth - 12\tabcolsep) * \real{0.1231}}
  >{\raggedright\arraybackslash}p{(\columnwidth - 12\tabcolsep) * \real{0.2769}}
  >{\raggedleft\arraybackslash}p{(\columnwidth - 12\tabcolsep) * \real{0.1692}}
  >{\raggedleft\arraybackslash}p{(\columnwidth - 12\tabcolsep) * \real{0.1231}}
  >{\raggedleft\arraybackslash}p{(\columnwidth - 12\tabcolsep) * \real{0.0462}}
  >{\raggedleft\arraybackslash}p{(\columnwidth - 12\tabcolsep) * \real{0.1231}}
  >{\centering\arraybackslash}p{(\columnwidth - 12\tabcolsep) * \real{0.1385}}@{}}
\caption{Comparaisons post-hoc Tukey par période. * p \textless{} 0,05 ;
** p \textless{} 0,01 ; *** p \textless{} 0,001.}\tabularnewline
\toprule\noalign{}
\begin{minipage}[b]{\linewidth}\raggedright
Période
\end{minipage} & \begin{minipage}[b]{\linewidth}\raggedright
Contraste
\end{minipage} & \begin{minipage}[b]{\linewidth}\raggedleft
Différence
\end{minipage} & \begin{minipage}[b]{\linewidth}\raggedleft
ESM
\end{minipage} & \begin{minipage}[b]{\linewidth}\raggedleft
t
\end{minipage} & \begin{minipage}[b]{\linewidth}\raggedleft
p-value
\end{minipage} & \begin{minipage}[b]{\linewidth}\centering
Signif.
\end{minipage} \\
\midrule\noalign{}
\endfirsthead
\toprule\noalign{}
\begin{minipage}[b]{\linewidth}\raggedright
Période
\end{minipage} & \begin{minipage}[b]{\linewidth}\raggedright
Contraste
\end{minipage} & \begin{minipage}[b]{\linewidth}\raggedleft
Différence
\end{minipage} & \begin{minipage}[b]{\linewidth}\raggedleft
ESM
\end{minipage} & \begin{minipage}[b]{\linewidth}\raggedleft
t
\end{minipage} & \begin{minipage}[b]{\linewidth}\raggedleft
p-value
\end{minipage} & \begin{minipage}[b]{\linewidth}\centering
Signif.
\end{minipage} \\
\midrule\noalign{}
\endhead
\bottomrule\noalign{}
\endlastfoot
& Ma Var1 - Ca Var1 & T1 & -7.761 & 35 & -2.00 & 0.3646 \\
& Ma Var1 - An Var1 & T1 & 2.705 & 35 & 0.62 & 0.9886 \\
* & Ma Var1 - Ma Var2 & T1 & -15.525 & 35 & -3.29 & 0.0256 \\
& Ma Var1 - Ca Var2 & T1 & -12.308 & 35 & -3.01 & 0.0501 \\
* & Ma Var1 - An Var2 & T1 & -14.004 & 35 & -3.60 & 0.0115 \\
& Ca Var1 - An Var1 & T1 & 10.466 & 35 & 2.19 & 0.2693 \\
& Ca Var1 - Ma Var2 & T1 & -7.764 & 35 & -1.52 & 0.6568 \\
& Ca Var1 - Ca Var2 & T1 & -4.548 & 35 & -1.00 & 0.9148 \\
& Ca Var1 - An Var2 & T1 & -6.243 & 35 & -1.43 & 0.7095 \\
* & An Var1 - Ma Var2 & T1 & -18.230 & 35 & -3.33 & 0.0234 \\
* & An Var1 - Ca Var2 & T1 & -15.013 & 35 & -3.03 & 0.0477 \\
* & An Var1 - An Var2 & T1 & -16.709 & 35 & -3.49 & 0.0153 \\
& Ma Var2 - Ca Var2 & T1 & 3.217 & 35 & 0.61 & 0.9896 \\
& Ma Var2 - An Var2 & T1 & 1.521 & 35 & 0.30 & 0.9997 \\
& Ca Var2 - An Var2 & T1 & -1.695 & 35 & -0.37 & 0.9990 \\
& Ma Var1 - Ca Var1 & T2 & -0.293 & 35 & -0.08 & 1.0000 \\
& Ma Var1 - An Var1 & T2 & 0.250 & 35 & 0.06 & 1.0000 \\
& Ma Var1 - Ma Var2 & T2 & 2.550 & 35 & 0.60 & 0.9907 \\
& Ma Var1 - Ca Var2 & T2 & -1.833 & 35 & -0.49 & 0.9961 \\
& Ma Var1 - An Var2 & T2 & 0.907 & 35 & 0.26 & 0.9998 \\
& Ca Var1 - An Var1 & T2 & 0.543 & 35 & 0.12 & 1.0000 \\
& Ca Var1 - Ma Var2 & T2 & 2.843 & 35 & 0.61 & 0.9895 \\
& Ca Var1 - Ca Var2 & T2 & -1.540 & 35 & -0.37 & 0.9990 \\
& Ca Var1 - An Var2 & T2 & 1.200 & 35 & 0.30 & 0.9996 \\
& An Var1 - Ma Var2 & T2 & 2.300 & 35 & 0.46 & 0.9971 \\
& An Var1 - Ca Var2 & T2 & -2.083 & 35 & -0.46 & 0.9971 \\
& An Var1 - An Var2 & T2 & 0.657 & 35 & 0.15 & 1.0000 \\
& Ma Var2 - Ca Var2 & T2 & -4.383 & 35 & -0.92 & 0.9399 \\
& Ma Var2 - An Var2 & T2 & -1.643 & 35 & -0.35 & 0.9992 \\
& Ca Var2 - An Var2 & T2 & 2.740 & 35 & 0.66 & 0.9847 \\
& Ma Var1 - Ca Var1 & T3 & -6.010 & 35 & -1.75 & 0.5072 \\
& Ma Var1 - An Var1 & T3 & 1.853 & 35 & 0.48 & 0.9965 \\
& Ma Var1 - Ma Var2 & T3 & -2.092 & 35 & -0.50 & 0.9957 \\
& Ma Var1 - Ca Var2 & T3 & -6.633 & 35 & -1.84 & 0.4536 \\
& Ma Var1 - An Var2 & T3 & 1.162 & 35 & 0.34 & 0.9994 \\
& Ca Var1 - An Var1 & T3 & 7.863 & 35 & 1.86 & 0.4402 \\
& Ca Var1 - Ma Var2 & T3 & 3.918 & 35 & 0.87 & 0.9517 \\
& Ca Var1 - Ca Var2 & T3 & -0.624 & 35 & -0.16 & 1.0000 \\
& Ca Var1 - An Var2 & T3 & 7.171 & 35 & 1.86 & 0.4412 \\
& An Var1 - Ma Var2 & T3 & -3.945 & 35 & -0.82 & 0.9626 \\
& An Var1 - Ca Var2 & T3 & -8.487 & 35 & -1.95 & 0.3930 \\
& An Var1 - An Var2 & T3 & -0.691 & 35 & -0.16 & 1.0000 \\
& Ma Var2 - Ca Var2 & T3 & -4.542 & 35 & -0.98 & 0.9223 \\
& Ma Var2 - An Var2 & T3 & 3.254 & 35 & 0.72 & 0.9781 \\
& Ca Var2 - An Var2 & T3 & 7.795 & 35 & 1.94 & 0.3933 \\
& Ma Var1 - Ca Var1 & T4 & 1.945 & 35 & 0.50 & 0.9959 \\
& Ma Var1 - An Var1 & T4 & 1.357 & 35 & 0.31 & 0.9996 \\
& Ma Var1 - Ma Var2 & T4 & -2.708 & 35 & -0.57 & 0.9923 \\
& Ma Var1 - Ca Var2 & T4 & -4.333 & 35 & -1.06 & 0.8950 \\
& Ma Var1 - An Var2 & T4 & 2.988 & 35 & 0.77 & 0.9715 \\
& Ca Var1 - An Var1 & T4 & -0.589 & 35 & -0.12 & 1.0000 \\
& Ca Var1 - Ma Var2 & T4 & -4.654 & 35 & -0.90 & 0.9426 \\
& Ca Var1 - Ca Var2 & T4 & -6.279 & 35 & -1.38 & 0.7409 \\
& Ca Var1 - An Var2 & T4 & 1.043 & 35 & 0.24 & 0.9999 \\
& An Var1 - Ma Var2 & T4 & -4.065 & 35 & -0.74 & 0.9756 \\
& An Var1 - Ca Var2 & T4 & -5.690 & 35 & -1.15 & 0.8589 \\
& An Var1 - An Var2 & T4 & 1.631 & 35 & 0.34 & 0.9993 \\
& Ma Var2 - Ca Var2 & T4 & -1.625 & 35 & -0.31 & 0.9996 \\
& Ma Var2 - An Var2 & T4 & 5.696 & 35 & 1.11 & 0.8749 \\
& Ca Var2 - An Var2 & T4 & 7.321 & 35 & 1.60 & 0.6015 \\
\end{longtable}

À la période T1, plusieurs contrastes sont significatifs et indiquent un
net avantage de la variété~2 pour certains fertilisants, alors qu'aux
périodes T2, T3 et T4 aucune comparaison fertilisant × variété n'est
significative, ce qui suggère un rapprochement des hauteurs entre
combinaisons au fil du temps.

\bookmarksetup{startatroot}

\chapter{Modèles ANOVA à mesures répétées (structure type
DABIRE)}\label{moduxe8les-anova-uxe0-mesures-ruxe9puxe9tuxe9es-structure-type-dabire}

Dans cette section, on implémente \textbf{exactement le même type de
modèles} que dans le rapport de DABIRE (ANOVA par période, puis ANOVA à
mesures répétées avec \texttt{car::Anova}), en utilisant directement les
variables \texttt{fertilisant}, \texttt{variete}, \texttt{periode} et
\texttt{hauteur} de la base.

\begin{Shaded}
\begin{Highlighting}[]
\FunctionTok{library}\NormalTok{(tidyverse)}
\FunctionTok{library}\NormalTok{(readxl)}
\FunctionTok{library}\NormalTok{(knitr)}
\ControlFlowTok{if}\NormalTok{ (}\SpecialCharTok{!}\FunctionTok{requireNamespace}\NormalTok{(}\StringTok{"nortest"}\NormalTok{, }\AttributeTok{quietly =} \ConstantTok{TRUE}\NormalTok{)) }\FunctionTok{install.packages}\NormalTok{(}\StringTok{"nortest"}\NormalTok{, }\AttributeTok{repos =} \StringTok{"https://cloud.r{-}project.org"}\NormalTok{, }\AttributeTok{quiet =} \ConstantTok{TRUE}\NormalTok{)}
\FunctionTok{library}\NormalTok{(nortest)}
\FunctionTok{library}\NormalTok{(rstatix)}
\FunctionTok{library}\NormalTok{(ggpubr)}
\FunctionTok{library}\NormalTok{(car)}
\end{Highlighting}
\end{Shaded}

\begin{Shaded}
\begin{Highlighting}[]
\CommentTok{\# Import des mêmes données que précédemment}
\NormalTok{d2 }\OtherTok{\textless{}{-}} \FunctionTok{read\_excel}\NormalTok{(}\StringTok{"donnees/donnees.xlsx"}\NormalTok{)}

\CommentTok{\# Conversion des virgules décimales en points et mise en forme des facteurs}
\NormalTok{d2}\SpecialCharTok{$}\NormalTok{hauteur }\OtherTok{\textless{}{-}} \FunctionTok{as.numeric}\NormalTok{(}\FunctionTok{gsub}\NormalTok{(}\StringTok{","}\NormalTok{, }\StringTok{"."}\NormalTok{, }\FunctionTok{as.character}\NormalTok{(d2}\SpecialCharTok{$}\NormalTok{hauteur)))}

\NormalTok{d2 }\OtherTok{\textless{}{-}}\NormalTok{ d2 }\SpecialCharTok{\%\textgreater{}\%}
  \FunctionTok{mutate}\NormalTok{(}
    \AttributeTok{fertilisant =} \FunctionTok{factor}\NormalTok{(fertilisant),               }\CommentTok{\# Ma, Ca, An}
    \AttributeTok{variete     =} \FunctionTok{factor}\NormalTok{(variete),                   }\CommentTok{\# Var1, Var2}
    \AttributeTok{periode     =} \FunctionTok{factor}\NormalTok{(periode, }\AttributeTok{levels =} \FunctionTok{c}\NormalTok{(}\StringTok{"T1"}\NormalTok{,}\StringTok{"T2"}\NormalTok{,}\StringTok{"T3"}\NormalTok{,}\StringTok{"T4"}\NormalTok{))}
\NormalTok{  )}
\end{Highlighting}
\end{Shaded}

\section{Modélisation : ANOVA par période + modèle à mesures
répétées}\label{moduxe9lisation-anova-par-puxe9riode-moduxe8le-uxe0-mesures-ruxe9puxe9tuxe9es}

\begin{Shaded}
\begin{Highlighting}[]
\CommentTok{\# Création d\textquotesingle{}IDs artificiels par (fertilisant, variete, periode)}
\NormalTok{d2\_id }\OtherTok{\textless{}{-}}\NormalTok{ d2 }\SpecialCharTok{\%\textgreater{}\%}
  \FunctionTok{group\_by}\NormalTok{(fertilisant, variete, periode) }\SpecialCharTok{\%\textgreater{}\%}
  \FunctionTok{arrange}\NormalTok{(No, }\AttributeTok{.by\_group =} \ConstantTok{TRUE}\NormalTok{) }\SpecialCharTok{\%\textgreater{}\%}
  \FunctionTok{mutate}\NormalTok{(}\AttributeTok{rang =} \FunctionTok{row\_number}\NormalTok{()) }\SpecialCharTok{\%\textgreater{}\%}
  \FunctionTok{ungroup}\NormalTok{() }\SpecialCharTok{\%\textgreater{}\%}
  \FunctionTok{mutate}\NormalTok{(}\AttributeTok{id =} \FunctionTok{interaction}\NormalTok{(fertilisant, variete, rang, }\AttributeTok{drop =} \ConstantTok{TRUE}\NormalTok{))}

\CommentTok{\# Passage au format large : une ligne = un "sujet", colonnes t1..t4}
\NormalTok{base }\OtherTok{\textless{}{-}}\NormalTok{ d2\_id }\SpecialCharTok{\%\textgreater{}\%}
  \FunctionTok{mutate}\NormalTok{(}
    \AttributeTok{periode\_mesure =}\NormalTok{ dplyr}\SpecialCharTok{::}\FunctionTok{recode}\NormalTok{(}
      \FunctionTok{as.character}\NormalTok{(periode),}
      \StringTok{"T1"} \OtherTok{=} \StringTok{"t1"}\NormalTok{,}
      \StringTok{"T2"} \OtherTok{=} \StringTok{"t2"}\NormalTok{,}
      \StringTok{"T3"} \OtherTok{=} \StringTok{"t3"}\NormalTok{,}
      \StringTok{"T4"} \OtherTok{=} \StringTok{"t4"}
\NormalTok{    )}
\NormalTok{  ) }\SpecialCharTok{\%\textgreater{}\%}
  \FunctionTok{select}\NormalTok{(id, fertilisant, variete, periode\_mesure, hauteur) }\SpecialCharTok{\%\textgreater{}\%}
\NormalTok{  tidyr}\SpecialCharTok{::}\FunctionTok{pivot\_wider}\NormalTok{(}
    \AttributeTok{names\_from  =}\NormalTok{ periode\_mesure,}
    \AttributeTok{values\_from =}\NormalTok{ hauteur}
\NormalTok{  ) }\SpecialCharTok{\%\textgreater{}\%}
  \FunctionTok{arrange}\NormalTok{(id)}

\CommentTok{\# Matrice de réponses hauteur (t1..t4) et facteurs}
\NormalTok{base}\SpecialCharTok{$}\NormalTok{hauteur    }\OtherTok{\textless{}{-}} \FunctionTok{as.matrix}\NormalTok{(base[, }\FunctionTok{c}\NormalTok{(}\StringTok{"t1"}\NormalTok{,}\StringTok{"t2"}\NormalTok{,}\StringTok{"t3"}\NormalTok{,}\StringTok{"t4"}\NormalTok{)])}
\NormalTok{base}\SpecialCharTok{$}\NormalTok{fertilisant }\OtherTok{\textless{}{-}} \FunctionTok{factor}\NormalTok{(base}\SpecialCharTok{$}\NormalTok{fertilisant)}
\NormalTok{base}\SpecialCharTok{$}\NormalTok{variete     }\OtherTok{\textless{}{-}} \FunctionTok{factor}\NormalTok{(base}\SpecialCharTok{$}\NormalTok{variete)}

\FunctionTok{attach}\NormalTok{(base)}
\end{Highlighting}
\end{Shaded}

\subsection{1. Effet du fertilisant à chaque
période}\label{effet-du-fertilisant-uxe0-chaque-puxe9riode}

\begin{Shaded}
\begin{Highlighting}[]
\NormalTok{mod.hauteur.fertilisant }\OtherTok{\textless{}{-}} \FunctionTok{lm}\NormalTok{(hauteur }\SpecialCharTok{\textasciitilde{}}\NormalTok{ fertilisant)}
\NormalTok{H\_fert }\OtherTok{\textless{}{-}} \FunctionTok{summary}\NormalTok{(}\FunctionTok{aov}\NormalTok{(mod.hauteur.fertilisant))}
\NormalTok{H\_fert}
\end{Highlighting}
\end{Shaded}

\begin{verbatim}
 Response t1 :
            Df Sum Sq Mean Sq F value Pr(>F)
fertilisant  2  258.3  129.15  1.2451 0.2994
Residuals   38 3941.9  103.73               

 Response t2 :
            Df  Sum Sq Mean Sq F value Pr(>F)
fertilisant  2   23.87  11.933   0.232 0.7941
Residuals   38 1954.41  51.432               

 Response t3 :
            Df  Sum Sq Mean Sq F value  Pr(>F)  
fertilisant  2  415.77 207.886  4.3114 0.02054 *
Residuals   38 1832.28  48.218                  
---
Signif. codes:  0 '***' 0.001 '**' 0.01 '*' 0.05 '.' 0.1 ' ' 1

 Response t4 :
            Df  Sum Sq Mean Sq F value Pr(>F)
fertilisant  2   82.58  41.291  0.6244  0.541
Residuals   38 2513.09  66.134               

40 observations effacées parce que manquantes
\end{verbatim}

\begin{Shaded}
\begin{Highlighting}[]
\NormalTok{LM\_fert }\OtherTok{\textless{}{-}} \FunctionTok{summary}\NormalTok{(mod.hauteur.fertilisant)}
\NormalTok{LM\_fert}
\end{Highlighting}
\end{Shaded}

\begin{verbatim}
Response t1 :

Call:
lm(formula = t1 ~ fertilisant)

Residuals:
     Min       1Q   Median       3Q      Max 
-25.7667  -5.7667  -0.1063   6.2333  19.1154 

Coefficients:
              Estimate Std. Error t value Pr(>|t|)    
(Intercept)     34.267      2.940  11.655  4.1e-14 ***
fertilisantCa    2.818      4.077   0.691    0.494    
fertilisantMa   -3.160      3.889  -0.813    0.422    
---
Signif. codes:  0 '***' 0.001 '**' 0.01 '*' 0.05 '.' 0.1 ' ' 1

Residual standard error: 10.18 on 38 degrees of freedom
  (40 observations effacées parce que manquantes)
Multiple R-squared:  0.0615,    Adjusted R-squared:  0.0121 
F-statistic: 1.245 on 2 and 38 DF,  p-value: 0.2994


Response t2 :

Call:
lm(formula = t2 ~ fertilisant)

Residuals:
    Min      1Q  Median      3Q     Max 
-21.317  -2.654   1.688   4.483   9.988 

Coefficients:
               Estimate Std. Error t value Pr(>|t|)    
(Intercept)   35.016667   2.070261  16.914   <2e-16 ***
fertilisantCa  1.637179   2.870935   0.570    0.572    
fertilisantMa -0.004167   2.738697  -0.002    0.999    
---
Signif. codes:  0 '***' 0.001 '**' 0.01 '*' 0.05 '.' 0.1 ' ' 1

Residual standard error: 7.172 on 38 degrees of freedom
  (40 observations effacées parce que manquantes)
Multiple R-squared:  0.01206,   Adjusted R-squared:  -0.03993 
F-statistic: 0.232 on 2 and 38 DF,  p-value: 0.7941


Response t3 :

Call:
lm(formula = t3 ~ fertilisant)

Residuals:
     Min       1Q   Median       3Q      Max 
-16.5562  -3.3563   0.9437   4.9167  12.4167 

Coefficients:
              Estimate Std. Error t value Pr(>|t|)    
(Intercept)     26.083      2.005  13.012 1.41e-15 ***
fertilisantCa    7.747      2.780   2.787  0.00826 ** 
fertilisantMa    1.973      2.652   0.744  0.46145    
---
Signif. codes:  0 '***' 0.001 '**' 0.01 '*' 0.05 '.' 0.1 ' ' 1

Residual standard error: 6.944 on 38 degrees of freedom
  (40 observations effacées parce que manquantes)
Multiple R-squared:  0.1849,    Adjusted R-squared:  0.1421 
F-statistic: 4.311 on 2 and 38 DF,  p-value: 0.02054


Response t4 :

Call:
lm(formula = t4 ~ fertilisant)

Residuals:
    Min      1Q  Median      3Q     Max 
-19.369  -2.108   1.631   4.131  12.392 

Coefficients:
              Estimate Std. Error t value Pr(>|t|)    
(Intercept)     33.608      2.348  14.316   <2e-16 ***
fertilisantCa    3.261      3.256   1.002    0.323    
fertilisantMa    2.985      3.106   0.961    0.342    
---
Signif. codes:  0 '***' 0.001 '**' 0.01 '*' 0.05 '.' 0.1 ' ' 1

Residual standard error: 8.132 on 38 degrees of freedom
  (40 observations effacées parce que manquantes)
Multiple R-squared:  0.03182,   Adjusted R-squared:  -0.01914 
F-statistic: 0.6244 on 2 and 38 DF,  p-value: 0.541
\end{verbatim}

Ces analyses univariées montrent que l'effet du \textbf{fertilisant} sur
la hauteur n'est significatif qu'à la période T3, où le fertilisant Ca
donne des plantes plus hautes que le fertilisant de référence, alors
qu'aux périodes T1, T2 et T4 les différences entre fertilisants ne sont
pas statistiquement détectables.

\subsection{2. Effet de la variété à chaque
période}\label{effet-de-la-variuxe9tuxe9-uxe0-chaque-puxe9riode}

\begin{Shaded}
\begin{Highlighting}[]
\NormalTok{mod.hauteur.variete }\OtherTok{\textless{}{-}} \FunctionTok{lm}\NormalTok{(hauteur }\SpecialCharTok{\textasciitilde{}}\NormalTok{ variete)}
\NormalTok{H\_var }\OtherTok{\textless{}{-}} \FunctionTok{summary}\NormalTok{(}\FunctionTok{aov}\NormalTok{(mod.hauteur.variete))}
\NormalTok{H\_var}
\end{Highlighting}
\end{Shaded}

\begin{verbatim}
 Response t1 :
            Df Sum Sq Mean Sq F value    Pr(>F)    
variete      1 1448.1 1448.12  20.522 5.447e-05 ***
Residuals   39 2752.0   70.57                      
---
Signif. codes:  0 '***' 0.001 '**' 0.01 '*' 0.05 '.' 0.1 ' ' 1

 Response t2 :
            Df  Sum Sq Mean Sq F value Pr(>F)
variete      1    1.29   1.288  0.0254 0.8742
Residuals   39 1976.98  50.692               

 Response t3 :
            Df  Sum Sq Mean Sq F value Pr(>F)
variete      1    9.72   9.718  0.1693  0.683
Residuals   39 2238.34  57.393               

 Response t4 :
            Df  Sum Sq Mean Sq F value Pr(>F)
variete      1   31.75  31.752   0.483 0.4912
Residuals   39 2563.92  65.741               

40 observations effacées parce que manquantes
\end{verbatim}

\begin{Shaded}
\begin{Highlighting}[]
\NormalTok{LM\_var }\OtherTok{\textless{}{-}} \FunctionTok{summary}\NormalTok{(mod.hauteur.variete)}
\NormalTok{LM\_var}
\end{Highlighting}
\end{Shaded}

\begin{verbatim}
Response t1 :

Call:
lm(formula = t1 ~ variete)

Residuals:
    Min      1Q  Median      3Q     Max 
-20.425  -5.488   2.012   5.012  15.212 

Coefficients:
            Estimate Std. Error t value Pr(>|t|)    
(Intercept)   28.925      1.715   16.87  < 2e-16 ***
varieteVar2   12.063      2.663    4.53 5.45e-05 ***
---
Signif. codes:  0 '***' 0.001 '**' 0.01 '*' 0.05 '.' 0.1 ' ' 1

Residual standard error: 8.4 on 39 degrees of freedom
  (40 observations effacées parce que manquantes)
Multiple R-squared:  0.3448,    Adjusted R-squared:  0.328 
F-statistic: 20.52 on 1 and 39 DF,  p-value: 5.447e-05


Response t2 :

Call:
lm(formula = t2 ~ variete)

Residuals:
    Min      1Q  Median      3Q     Max 
-21.623  -2.324   1.677   4.317   9.877 

Coefficients:
            Estimate Std. Error t value Pr(>|t|)    
(Intercept)  35.6833     1.4533  24.553   <2e-16 ***
varieteVar2  -0.3598     2.2570  -0.159    0.874    
---
Signif. codes:  0 '***' 0.001 '**' 0.01 '*' 0.05 '.' 0.1 ' ' 1

Residual standard error: 7.12 on 39 degrees of freedom
  (40 observations effacées parce que manquantes)
Multiple R-squared:  0.0006512, Adjusted R-squared:  -0.02497 
F-statistic: 0.02541 on 1 and 39 DF,  p-value: 0.8742


Response t3 :

Call:
lm(formula = t3 ~ variete)

Residuals:
   Min     1Q Median     3Q    Max 
-17.40  -3.40   1.60   4.50  12.11 

Coefficients:
            Estimate Std. Error t value Pr(>|t|)    
(Intercept)  28.9000     1.5464  18.688   <2e-16 ***
varieteVar2   0.9882     2.4016   0.411    0.683    
---
Signif. codes:  0 '***' 0.001 '**' 0.01 '*' 0.05 '.' 0.1 ' ' 1

Residual standard error: 7.576 on 39 degrees of freedom
  (40 observations effacées parce que manquantes)
Multiple R-squared:  0.004323,  Adjusted R-squared:  -0.02121 
F-statistic: 0.1693 on 1 and 39 DF,  p-value: 0.683


Response t4 :

Call:
lm(formula = t4 ~ variete)

Residuals:
    Min      1Q  Median      3Q     Max 
-17.567  -2.567   1.933   5.147  11.233 

Coefficients:
            Estimate Std. Error t value Pr(>|t|)    
(Intercept)   35.067      1.655  21.188   <2e-16 ***
varieteVar2    1.786      2.570   0.695    0.491    
---
Signif. codes:  0 '***' 0.001 '**' 0.01 '*' 0.05 '.' 0.1 ' ' 1

Residual standard error: 8.108 on 39 degrees of freedom
  (40 observations effacées parce que manquantes)
Multiple R-squared:  0.01223,   Adjusted R-squared:  -0.01309 
F-statistic: 0.483 on 1 and 39 DF,  p-value: 0.4912
\end{verbatim}

On observe un effet très net de la \textbf{variété} à T1, où la
variété~2 produit des plantes sensiblement plus hautes que la variété~1,
alors qu'aux périodes T2, T3 et T4 les hauteurs des deux variétés
deviennent statistiquement similaires.

\subsection{3. Effet d'interaction variété × fertilisant à chaque
période}\label{effet-dinteraction-variuxe9tuxe9-fertilisant-uxe0-chaque-puxe9riode}

\begin{Shaded}
\begin{Highlighting}[]
\NormalTok{mod.hauteur.interaction }\OtherTok{\textless{}{-}} \FunctionTok{lm}\NormalTok{(hauteur }\SpecialCharTok{\textasciitilde{}}\NormalTok{ variete }\SpecialCharTok{*}\NormalTok{ fertilisant)}
\NormalTok{H\_int }\OtherTok{\textless{}{-}} \FunctionTok{summary}\NormalTok{(}\FunctionTok{aov}\NormalTok{(mod.hauteur.interaction))}
\NormalTok{H\_int}
\end{Highlighting}
\end{Shaded}

\begin{verbatim}
 Response t1 :
                    Df  Sum Sq Mean Sq F value   Pr(>F)    
variete              1 1448.12 1448.12 21.6812 4.51e-05 ***
fertilisant          2  132.98   66.49  0.9955   0.3798    
variete:fertilisant  2  281.37  140.68  2.1063   0.1368    
Residuals           35 2337.70   66.79                     
---
Signif. codes:  0 '***' 0.001 '**' 0.01 '*' 0.05 '.' 0.1 ' ' 1

 Response t2 :
                    Df  Sum Sq Mean Sq F value Pr(>F)
variete              1    1.29   1.288  0.0234 0.8793
fertilisant          2   24.88  12.440  0.2261 0.7988
variete:fertilisant  2   26.13  13.065  0.2374 0.7899
Residuals           35 1925.97  55.028               

 Response t3 :
                    Df  Sum Sq Mean Sq F value  Pr(>F)  
variete              1    9.72   9.718  0.1873 0.66787  
fertilisant          2  417.67 208.834  4.0238 0.02674 *
variete:fertilisant  2    4.16   2.082  0.0401 0.96073  
Residuals           35 1816.51  51.900                  
---
Signif. codes:  0 '***' 0.001 '**' 0.01 '*' 0.05 '.' 0.1 ' ' 1

 Response t4 :
                    Df  Sum Sq Mean Sq F value Pr(>F)
variete              1   31.75  31.752  0.4717 0.4967
fertilisant          2  111.98  55.991  0.8318 0.4437
variete:fertilisant  2   95.97  47.987  0.7129 0.4972
Residuals           35 2355.96  67.313               

40 observations effacées parce que manquantes
\end{verbatim}

L'interaction \textbf{variété × fertilisant} n'est significative à
aucune période, ce qui suggère que, à instant donné, l'effet de la
variété est globalement parallèle d'un fertilisant à l'autre.

\subsection{\texorpdfstring{4. Modèle global à mesures répétées avec
\texttt{car::Anova}}{4. Modèle global à mesures répétées avec car::Anova}}\label{moduxe8le-global-uxe0-mesures-ruxe9puxe9tuxe9es-avec-caranova}

\begin{Shaded}
\begin{Highlighting}[]
\NormalTok{fact.temps }\OtherTok{\textless{}{-}} \FunctionTok{data.frame}\NormalTok{(}\AttributeTok{TEMPS =} \FunctionTok{as.factor}\NormalTok{(}\DecValTok{1}\SpecialCharTok{:}\DecValTok{4}\NormalTok{))}

\NormalTok{mod.hauteur.tps.interaction }\OtherTok{\textless{}{-}} \FunctionTok{Anova}\NormalTok{(}
\NormalTok{  mod.hauteur.interaction,}
  \AttributeTok{idata   =}\NormalTok{ fact.temps,}
  \AttributeTok{idesign =} \SpecialCharTok{\textasciitilde{}}\NormalTok{ TEMPS,}
  \AttributeTok{type    =} \StringTok{"III"}\NormalTok{,}
  \AttributeTok{test    =} \StringTok{"Wilks"}
\NormalTok{)}

\FunctionTok{summary}\NormalTok{(mod.hauteur.tps.interaction)}
\end{Highlighting}
\end{Shaded}

\begin{verbatim}

Type III Repeated Measures MANOVA Tests:

------------------------------------------
 
Term: (Intercept) 

 Response transformation matrix:
   (Intercept)
t1           1
t2           1
t3           1
t4           1

Sum of squares and products for the hypothesis:
            (Intercept)
(Intercept)    72192.13

Multivariate Tests: (Intercept)
                 Df test stat approx F num Df den Df     Pr(>F)    
Pillai            1  0.910358 355.4402      1     35 < 2.22e-16 ***
Wilks             1  0.089642 355.4402      1     35 < 2.22e-16 ***
Hotelling-Lawley  1 10.155434 355.4402      1     35 < 2.22e-16 ***
Roy               1 10.155434 355.4402      1     35 < 2.22e-16 ***
---
Signif. codes:  0 '***' 0.001 '**' 0.01 '*' 0.05 '.' 0.1 ' ' 1

------------------------------------------
 
Term: variete 

 Response transformation matrix:
   (Intercept)
t1           1
t2           1
t3           1
t4           1

Sum of squares and products for the hypothesis:
            (Intercept)
(Intercept)    666.0362

Multivariate Tests: variete
                 Df test stat approx F num Df den Df   Pr(>F)  
Pillai            1 0.0856665  3.27925      1     35 0.078752 .
Wilks             1 0.9143335  3.27925      1     35 0.078752 .
Hotelling-Lawley  1 0.0936929  3.27925      1     35 0.078752 .
Roy               1 0.0936929  3.27925      1     35 0.078752 .
---
Signif. codes:  0 '***' 0.001 '**' 0.01 '*' 0.05 '.' 0.1 ' ' 1

------------------------------------------
 
Term: fertilisant 

 Response transformation matrix:
   (Intercept)
t1           1
t2           1
t3           1
t4           1

Sum of squares and products for the hypothesis:
            (Intercept)
(Intercept)    1096.433

Multivariate Tests: fertilisant
                 Df test stat approx F num Df den Df   Pr(>F)  
Pillai            2 0.1336274 2.699162      2     35 0.081252 .
Wilks             2 0.8663726 2.699162      2     35 0.081252 .
Hotelling-Lawley  2 0.1542378 2.699162      2     35 0.081252 .
Roy               2 0.1542378 2.699162      2     35 0.081252 .
---
Signif. codes:  0 '***' 0.001 '**' 0.01 '*' 0.05 '.' 0.1 ' ' 1

------------------------------------------
 
Term: variete:fertilisant 

 Response transformation matrix:
   (Intercept)
t1           1
t2           1
t3           1
t4           1

Sum of squares and products for the hypothesis:
            (Intercept)
(Intercept)    35.67558

Multivariate Tests: variete:fertilisant
                 Df test stat   approx F num Df den Df  Pr(>F)
Pillai            2 0.0049935 0.08782492      2     35 0.91612
Wilks             2 0.9950065 0.08782492      2     35 0.91612
Hotelling-Lawley  2 0.0050186 0.08782492      2     35 0.91612
Roy               2 0.0050186 0.08782492      2     35 0.91612

------------------------------------------
 
Term: TEMPS 

 Response transformation matrix:
   TEMPS1 TEMPS2 TEMPS3
t1      1      0      0
t2      0      1      0
t3      0      0      1
t4     -1     -1     -1

Sum of squares and products for the hypothesis:
        TEMPS1  TEMPS2  TEMPS3
TEMPS1 504.008 -42.168 445.776
TEMPS2 -42.168   3.528 -37.296
TEMPS3 445.776 -37.296 394.272

Multivariate Tests: TEMPS
                 Df test stat approx F num Df den Df   Pr(>F)  
Pillai            1 0.2300553  3.28674      3     33 0.032752 *
Wilks             1 0.7699447  3.28674      3     33 0.032752 *
Hotelling-Lawley  1 0.2987946  3.28674      3     33 0.032752 *
Roy               1 0.2987946  3.28674      3     33 0.032752 *
---
Signif. codes:  0 '***' 0.001 '**' 0.01 '*' 0.05 '.' 0.1 ' ' 1

------------------------------------------
 
Term: variete:TEMPS 

 Response transformation matrix:
   TEMPS1 TEMPS2 TEMPS3
t1      1      0      0
t2      0      1      0
t3      0      0      1
t4     -1     -1     -1

Sum of squares and products for the hypothesis:
          TEMPS1    TEMPS2     TEMPS3
TEMPS1 981.03717 52.116167 124.253500
TEMPS2  52.11617  2.768595   6.600786
TEMPS3 124.25350  6.600786  15.737357

Multivariate Tests: variete:TEMPS
                 Df test stat approx F num Df den Df   Pr(>F)  
Pillai            1 0.2275214 3.239876      3     33 0.034423 *
Wilks             1 0.7724786 3.239876      3     33 0.034423 *
Hotelling-Lawley  1 0.2945342 3.239876      3     33 0.034423 *
Roy               1 0.2945342 3.239876      3     33 0.034423 *
---
Signif. codes:  0 '***' 0.001 '**' 0.01 '*' 0.05 '.' 0.1 ' ' 1

------------------------------------------
 
Term: fertilisant:TEMPS 

 Response transformation matrix:
   TEMPS1 TEMPS2 TEMPS3
t1      1      0      0
t2      0      1      0
t3      0      0      1
t4     -1     -1     -1

Sum of squares and products for the hypothesis:
          TEMPS1   TEMPS2    TEMPS3
TEMPS1 512.46860 90.53914 408.14814
TEMPS2  90.53914 22.46038  74.88305
TEMPS3 408.14814 74.88305 326.25438

Multivariate Tests: fertilisant:TEMPS
                 Df test stat approx F num Df den Df  Pr(>F)
Pillai            2 0.1644683 1.015495      6     68 0.42274
Wilks             2 0.8358609 1.031664      6     66 0.41287
Hotelling-Lawley  2 0.1959775 1.045213      6     64 0.40482
Roy               2 0.1939470 2.198066      3     34 0.10623

------------------------------------------
 
Term: variete:fertilisant:TEMPS 

 Response transformation matrix:
   TEMPS1 TEMPS2 TEMPS3
t1      1      0      0
t2      0      1      0
t3      0      0      1
t4     -1     -1     -1

Sum of squares and products for the hypothesis:
         TEMPS1    TEMPS2   TEMPS3
TEMPS1 668.4031 143.01790 258.1318
TEMPS2 143.0179  71.05914  61.6922
TEMPS3 258.1318  61.69220 100.7198

Multivariate Tests: variete:fertilisant:TEMPS
                 Df test stat  approx F num Df den Df  Pr(>F)
Pillai            2 0.1508520 0.9245641      6     68 0.48303
Wilks             2 0.8505399 0.9273876      6     66 0.48129
Hotelling-Lawley  2 0.1740872 0.9284653      6     64 0.48078
Roy               2 0.1641155 1.8599762      3     34 0.15499

Univariate Type III Repeated-Measures ANOVA Assuming Sphericity

                           Sum Sq num Df Error SS den Df  F value  Pr(>F)    
(Intercept)               18048.0      1   1777.2     35 355.4402 < 2e-16 ***
variete                     166.5      1   1777.2     35   3.2792 0.07875 .  
fertilisant                 274.1      2   1777.2     35   2.6992 0.08125 .  
variete:fertilisant           8.9      2   1777.2     35   0.0878 0.91612    
TEMPS                       493.2      3   6659.0    105   2.5923 0.05660 .  
variete:TEMPS               658.2      3   6659.0    105   3.4594 0.01905 *  
fertilisant:TEMPS           359.1      6   6659.0    105   0.9437 0.46720    
variete:fertilisant:TEMPS   398.7      6   6659.0    105   1.0478 0.39889    
---
Signif. codes:  0 '***' 0.001 '**' 0.01 '*' 0.05 '.' 0.1 ' ' 1


Mauchly Tests for Sphericity

                          Test statistic p-value
TEMPS                            0.80217 0.19052
variete:TEMPS                    0.80217 0.19052
fertilisant:TEMPS                0.80217 0.19052
variete:fertilisant:TEMPS        0.80217 0.19052


Greenhouse-Geisser and Huynh-Feldt Corrections
 for Departure from Sphericity

                           GG eps Pr(>F[GG])  
TEMPS                     0.86184    0.06600 .
variete:TEMPS             0.86184    0.02503 *
fertilisant:TEMPS         0.86184    0.45889  
variete:fertilisant:TEMPS 0.86184    0.39574  
---
Signif. codes:  0 '***' 0.001 '**' 0.01 '*' 0.05 '.' 0.1 ' ' 1

                             HF eps Pr(>F[HF])
TEMPS                     0.9365994 0.06072900
variete:TEMPS             0.9365994 0.02159004
fertilisant:TEMPS         0.9365994 0.46356203
variete:fertilisant:TEMPS 0.9365994 0.39757284
\end{verbatim}

Les tests multivariés confirment un effet global du \textbf{temps} et
une interaction significative \textbf{variété × temps}, indiquant que
l'écart entre variétés varie au cours des quatre mesures, tandis que les
interactions impliquant le fertilisant avec le temps restent non
significatives.

\bookmarksetup{startatroot}

\chapter{Conclusion}\label{conclusion}

Cette \textbf{ANOVA à mesures répétées} (implémentée à la fois avec
\texttt{afex::aov\_car} et avec la structure multivariée de
\texttt{car::Anova}) permet de tester si les facteurs \textbf{terreau}
(\texttt{groupe} / \texttt{fertilisant}), \textbf{variété}
(\texttt{variete}) et \textbf{période} (\texttt{periode} /
\texttt{TEMPS}), ainsi que leurs interactions, expliquent la variable
dépendante \textbf{taille}. Après vérification des hypothèses (normalité
des résidus, homogénéité des variances, sphéricité et corrections
associées), l'interprétation des effets significatifs et des
\textbf{comparaisons post-hoc} permet d'identifier les combinaisons
terreau × variété et les périodes qui conduisent aux plus fortes
hauteurs de plantes.




\end{document}
